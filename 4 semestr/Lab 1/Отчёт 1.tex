\documentclass{article}
\usepackage[T2A]{fontenc}
\usepackage[utf8]{inputenc}
\usepackage[russian]{babel}
\usepackage{alltt}
\usepackage{amsmath}
\usepackage{amsfonts}
\usepackage{indentfirst}

\title{non-linear equations}
\author{Иван Золин}
\date{November 2021}
\thispagestyle{empty}
\begin{document}
	
	\large
	\begin{center}
		
		Санкт-Петербургский политехнический университет\\
		Высшая школа прикладной математики и вычислительной физики, ФизМех
		
		~\\
		~\\
		~\\
		~\\
		Направление подготовки\\
		«01.03.02 Прикладная математика и информатика»
		
		~\\
		~\\
		~\\
		~\\
		Отчет по лабораторной работе № 1\\
		\textbf{тема "Интерполяция табличных функций" }	
		~\\	Дисциплина "Численные методы"
	\end{center}
	
	~\\
	~\\
	~\\
	~\\
	~\\
	~\\
	~\\
	\begin{alltt}
		Выполнил студент гр. 5030102/00001			  		   		  Золин И.М.
		Преподаватель: 				              	        	Добрецова С.Б.
	\end{alltt}
	
	~\\
	~\\
	\begin{center}
		Санкт-Петербург
		
		~\\
		\textbf{202}
	\end{center}{}
	
	\newpage
	
	\section{Формулировка задачи и её формализация}
	\subsection{Формулировка задания}
	Задача интерполирования табличной функции – весьма важная задача, которую необходимо решать численными методами. Во-первых, нам может быть необходимо вычислить значения функции в промежуточных точках. Также для упрощения сложных вычислений мы должны заменять сложную функцию на функцию полинома для более простого численного интегрирования и дифференцирования функции. Более того, нужно обеспечить «близость» интерполяционного полинома к исходной сложной функции.
	\subsection{Постановка задания}
	Дана матрица $A \in \mathbb{R}^{n \times n}$. Если выполнено условие $A \cdot X = \lambda \cdot X$,где X - ненулевой вектор, $\lambda$ - это  число, то Х – это собственный вектор (СВ) матрицы А. $\lambda$ - это собственное число (СЧ) матрицы А, соответсвующее СВ X.
	
	$A \cdot X = \lambda \cdot X \Leftrightarrow \det(A-\lambda E) = 0 \Leftrightarrow p_0\lambda^n + p_1\lambda^{(n-1)}+...+p_n$
	
	Из основной теоремы алгебры следует, что матрица А имеет ровно n СЧ.
	
	Требуется решить частичную алгебраическую проблему собственных значений: найти два максимальных по модулю, но разных по знаку собственных числа, используя метод скалярных произведений.
	
	
	\section{Алгоритм метода и условия его применимости}   
	
	\subsection{Построение интерполяционного полинома в форме Лангранжа:}
	\subsubsection{Алгоритм}
	1. Входные данные: пусть х – аргумент интерполяционного полинома; xk,yk – сетка и сеточная функция;  
	
	2. Цикл:
	
	k = 0, i = 0, $basics = 0, res = 0$
	
	для i от 0 до size: 
	
	mul = 1.0;
	
		для k от 0 до size:
		
			если k не равно i:
				
				домножить basics на величину $frac{(x - xk[k])}{(xk[i]-xk[k])}$;
			
			конец «если»
		
		конец цикла
		
		прибавить к res величину yk[i] * basics
	
	конец цикла
	
	вернуть res;

	
	Конец
	
	3. Результат: $res$
	~\\
	
	Построение чебышевской сетки на выбранном интервале:
	
	$x^{(0)} \in R^n,y^{(1)} = A*x^{(1)} x^{(1)} = \frac{y^{(1)}}{\mu_1}, \mu_1 = ||x^{(1)}||$
	
	
	\subsubsection{Условия применимости метода}
	Пусть $A \in \mathbb{R}^{n \times n}$, а $\lambda_1, \lambda_2,...,\lambda_n$ - собственные числа, исследуемой матрицы A, причём $\lambda_1 = -\lambda_2$ и $|\lambda_1| = |\lambda_2| > |\lambda_3| \geq |\lambda_4| \geq ... \geq |\lambda_n|$, матрица А - матрица простой структуры и симметричная. Тогда метод скалярных произведений сходится.
	
	
	
	\section{Предварительный анализ задачи}
	Согласно теореме: если матрица вещественная и симметричная, то она является матрицей простой структуры. Будем задавать матрицы, удовлетворяющие данному критерию.
	
	Такие матрицы будем искать с помощью QR-разложения из пакета Matlab, где матрица Q – ортогональная, R - верхнетреугольная.(Генерируются матрица, полученную матрицу представляем в виде QR-разложения, т.е. умножения 2 матриц Q и R). С помощью QR-разложения находим матрицу Q.
	
	Задается диагональная матрица D, у нее на диагонали – собственные числа. 
	
	Нужная симметричная матрица получается следующим образом:
	$A = Q \cdot D \cdot Q^T$
	
	\section{Тестовый пример с детальными расчётами для задачи малой размерности}
	
	
	
	\[A =							
	\left(
	\begin{array}{ccc}
		-1.6693 & 3.5021 & 5.3492\\
		3.5021 & 3.4792 & 0.9209 \\
		5.3492 & 0.9209 & -0.8099
	\end{array}
	\right)
	\] 
	
	Точные собственные числа:
	$\lambda_{1,2} = \pm 7 , \lambda_3 = 1$
	
	Точные собственные векторы:
	
	$X_1 = \begin{pmatrix}
		0.7004\\
		0.1444\\
		0.7045\\
	\end{pmatrix}$,
	$X_2 =\begin{pmatrix}
		0.1500\\
		0.9414\\
		-0.3020
	\end{pmatrix}$,
	
	$X_3 =\begin{pmatrix}
		0.1500\\
		0.3173\\
		0.6422
	\end{pmatrix}$,
	
	%	Метод скалярных произведений:
	Начальные приближения:
	$x^{(-1)}=\begin{pmatrix}
		\frac{1}{\sqrt{3}}\\
		\frac{1}{\sqrt{3}}\\
		\frac{1}{\sqrt{3}}
	\end{pmatrix}$,
	$x^{(0)}=\begin{pmatrix}
		\frac{1}{\sqrt{3}}\\
		\frac{1}{\sqrt{3}}\\
		\frac{1}{\sqrt{3}}
	\end{pmatrix}$,
	~\\
	
	1-ая итерация:
	
	\[y^{(1)} = A \cdot x^{(0)}							
	\left(
	\begin{array}{ccc}
		4.1469 \\
		4.5627 \\
		3.1527
	\end{array}
	\right) \] 
	
	$\|y^{(1)}\| = 6.9249$
	
	$({y^{(1)}},{x^{(-1)}}) = 6.8493$
	
	$\lambda^{(1)} = \sqrt {{({y^{(1)}},{x^{(-1)}})} \cdot ||x^{(0)}||} =  2.6172$
	
	
	\[x^{(1)} = \frac{y^{(1)}}{||y^{(1)}||}= 							
	\left(
	\begin{array}{ccc}
		0.5988  \\
		0.6589  \\
		0.4553
	\end{array}
	\right) \] 
	
	\[Ax^{(1)} =							
	\left(
	\begin{array}{ccc}
		3.7432 \\
		4.8088 \\
		3.4413
	\end{array}
	\right) \]
	
	\[\lambda^{(1)} x^{(1)} =							
	\left(
	\begin{array}{ccc}
		1.5673 \\
		1.7245 \\
		1.1915
	\end{array}
	\right) \]
	\[Ax^{(1)} - \lambda^{(1)} x^{(1)} =							
	\left(
	\begin{array}{ccc}
		2.1759 \\
		3.0844 \\
		2.2498
	\end{array}
	\right) \]
	
	
	\[w^{(1)}_1 =x^{(1)} +  x^{(0)}=							
	\left(
	\begin{array}{ccc}
		1.1762 \\
		1.2363 \\
		1.0327
	\end{array}
	\right) \]
	
	\[w^{(1)}_2 =x^{(1)} -  x^{(0)}=							
	\left(
	\begin{array}{ccc}
		0.0214 \\
		0.0815 \\
		-0.1221 
	\end{array}
	\right) \]
	
	$||Ax^{(1)} - \lambda^{(2)} x^{(1)}|| = 4.3942$,
	$||w^{(1)}_1|| = 1.9946$
	
	$\frac{||Ax^{(1)} - \lambda^{(1)} x^{(1)}||}{||w^{(1)}_1||} = 2.2031$	
	
	
	~\\
	2-ая итерация:
	\[y^{(2)} = A \cdot x^{(1)}							
	\left(
	\begin{array}{ccc}
		3.7432 \\
		4.8088 \\
		3.4413
	\end{array}
	\right) \] 
	
	$||y^{(2)}|| = 6.9985$
	
	$({y^{(2)}},{x^{(0)}}) = 6.9249
	$
	
	$\lambda^{(2)} = \sqrt {{({y^{(2)}},{x^{(0)}})} \cdot ||y^{(1)}||} =  6.9249$
	
	\[x^{(2)} = \frac{y^{(2)}}{||y^{(2)}||}= 							
	\left(
	\begin{array}{ccc}
		0.5349 \\
		0.6871 \\
		0.4917
	\end{array}
	\right) \] 
	
	\[Ax^{(2)} =							
	\left(
	\begin{array}{ccc}
		4.1439 \\
		4.7166 \\
		3.0956
	\end{array}
	\right) \]
	
	\[\lambda^{(2)} x^{(2)} =							
	\left(
	\begin{array}{ccc}
		3.7038 \\
		4.7583 \\
		3.4052
	\end{array}
	\right) \]
	\[Ax^{(2)} - \lambda^{(2)} x^{(2)} =							
	\left(
	\begin{array}{ccc}
		0.4400 \\
		-0.0417 \\
		-0.3096
	\end{array}
	\right) \]
	
	\[w^{(2)}_1 = x^{(2)} +  x^{(1)}=							
	\left(
	\begin{array}{ccc}
		1.1337 \\
		1.3460 \\
		0.9470
	\end{array}
	\right) \]
	
	\[w^{(2)}_2 = x^{(2)} -  x^{(1)}=							
	\left(
	\begin{array}{ccc}
		-0.0640 \\
		0.0282 \\
		0.0365
	\end{array}
	\right) \]
	
	$||Ax^{(2)} - \lambda^{(2)} x^{(2)}|| = 0.5396$,
	$||w^{(2)}_1|| = 1.9984$
	
	$\frac{||Ax^{(2)} - \lambda^{(2)} x^{(2)}||}{||w^{(2)}_1||} = 0.2700$	
	~\\
	
	3-ья итерация:
	\[y^{(3)} = A \cdot x^{(2)}							
	\left(
	\begin{array}{ccc}
		4.1439 \\
		4.7166 \\
		3.0956
	\end{array}
	\right) \] 
	
	$||y^{(3)}|| = 7.000$
	
	$({y^{(3)}},{x^{(1)}}) = 6.9985$
	
	$\lambda^{(3)} = \sqrt {{({y^{(3)}},{x^{(1)}})} \cdot ||y^{(2)}||} =  6.9985$
	
	\[x^{(3)} = \frac{y^{(3)}}{||y^{(3)}||} =							
	\left(
	\begin{array}{ccc}
		0.5920 \\
		0.6738 \\
		0.4422
	\end{array}
	\right) \] 
	
	\[Ax^{(3)} =							
	\left(
	\begin{array}{ccc}
		3.7370 \\
		4.8247 \\
		3.4290
	\end{array}
	\right) \]
	\[\lambda^{(3)} x^{(3)} =							
	\left(
	\begin{array}{ccc}
		4.1430 \\
		4.7155 \\
		3.0949
	\end{array}
	\right) \]
	\[Ax^{(3)} - \lambda^{(3)} x^{(3)} =							
	\left(
	\begin{array}{ccc}
		-0.4059 \\
		0.1091 \\
		0.3341
	\end{array}
	\right) \]
	
	$||Ax^{(3)} - \lambda^{(3)} x^{(3)}|| = 0.5369$,
	\[w^{(3)}_1 = x^{(3)} +  x^{(2)}=							
	\left(
	\begin{array}{ccc}
		1.1268  \\
		1.3609  \\
		0.9340
	\end{array}
	\right) \]
	
	\[w^{(3)}_2 = x^{(3)} -  x^{(2)}=							
	\left(
	\begin{array}{ccc}
		0.0571  \\
		-0.0133  \\
		-0.0495
	\end{array}
	\right) \]
	
	$\frac{||Ax^{(3)} - \lambda^{(3)} x^{(3)}||}{||w^{(3)}_1||} = 0.2687$
	
	
	
	\section{Подготовка контрольных тестов}
	\paragraph{}Создаётся симметричная матрица  $A_{10 \times 10}$  для нахождения собственных чисел с помощью метода скалярных произведений с точностью  $\epsilon=10^{-i}$, где $i \in [0, 14], i \in \mathbb {N} $. 
	Собственные числа:[-10, 10, 1, 8, 2, 7, 6, 5, 3, 4]
	
	\section{Модульная структура программы}
	ScalarProductMethod(вх: $A, \epsilon$, вых: $\lambda_{1,2}, w_{1,2}, n$) Находит  максимальные собственные числа разные по знаку матрицы А  $\lambda_{1,2}$ и соответствующие им собственные вектора $w_{1,2}$ с помощью метода скалярных произведений за n итераций с заданной точностью $\epsilon$.
	
	
	\section{Анализ результатов}
	1.По графику зависимости количества итераций метода скалярных произведений от заданной точности видно, что количество итераций увеличивается линейно [4 5 7 9 11 14 16 19 24 29 34 39 44 50].
	При изменении $\epsilon$ на порядок количество итераций изменяется примерно на 20. 
	
	2. Графики заивисимости с.в. и невязок от заданной точности линейны.
	
	
	\section{Выводы}
	С помошью метода скалярных произведений достигается заданная точность нахождения собственных чисел, при хорошей отделимости.
	
\end{document}