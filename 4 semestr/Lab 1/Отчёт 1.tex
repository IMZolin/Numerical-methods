\documentclass{article}
\usepackage[T2A]{fontenc}
\usepackage[utf8]{inputenc}
\usepackage[russian]{babel}
\usepackage{alltt}
\usepackage{amsmath}
\usepackage{amsfonts}
\usepackage{indentfirst}

\title{non-linear equations}
\author{Иван Золин}
\date{April 2022}
\thispagestyle{empty}
\begin{document}
	
	\large
	\begin{center}
		
		Санкт-Петербургский политехнический университет\\
		Высшая школа прикладной математики и вычислительной физики, ФизМех
		
		~\\
		~\\
		~\\
		~\\
		Направление подготовки\\
		«01.03.02 Прикладная математика и информатика»
		
		~\\
		~\\
		~\\
		~\\
		Отчет по лабораторной работе № 1\\
		\textbf{тема "Интерполяция табличных функций" }	
		~\\	Дисциплина "Численные методы"
	\end{center}
	
	~\\
	~\\
	~\\
	~\\
	~\\
	~\\
	~\\
	\begin{alltt}
		Выполнил студент гр. 5030102/00001			  		   		  Золин И.М.
		Преподаватель: 				              	        	Добрецова С.Б.
	\end{alltt}
	
	~\\
	~\\
	\begin{center}
		Санкт-Петербург
		
		~\\
		\textbf{2022}
	\end{center}{}
	
	\newpage
	
	\section{Формулировка задачи и её формализация}
	\subsection{Формулировка задания}
	Нам дан набор точек $(x_i,y_i), i = 0,..., n$ и их количество (n+1).
	
	пусть $x^h = \left\{x_i\right\}^n_{i=0}$ - сетка, $y^h = \left\{y_i\right\}^n_{i=0}$ - сеточная функция. Пусть табличная функция задана парой элементов $(x^h,y^h)$. Требуется построить функцию $\phi(x)$ в форме интерполяционного полинома Лангранжа, которая удовлетворяет критерию близости:
	
	$\phi(x) \approx (x^h,y^h)$
	
	Для построения полинома необходимо использовать чебышевскую сетки.
	
	Также нужно исследовать влияние количества узлов на точность интерполяции и сходимость интерполяционного процесса.
	
	\section{Алгоритм метода и условия его применимости}   
	
	\subsection{Построение интерполяционного полинома в форме Лангранжа:}
	\subsubsection{Алгоритм}
	1. Входные данные: пусть х – аргумент интерполяционного полинома; $x^h,y^h$ – сетка и сеточная функция;  
	
	2. Цикл:
	
	k = 0, i = 0, $basics = 0, res = 0$
	
	для i от 0 до size: 
	
	\quad mul = 1.0;
	
	\quad для k от 0 до size:
		
	\qquad если k не равно i:
				
	\quad \qquad $basics *=\frac{(x - x^h_k)}{(x^h_i-x^h_k)}$;
			
	\qquad конец «если»
		
	\quad конец цикла
		
	\quad res += $y^h_i * basics$
	
	конец цикла
	
	вернуть res;

	
	Конец
	
	3. Результат: $res$
	~\\
	
	Построение чебышевской сетки на выбранном интервале:
	
	$x^{(0)} \in R^n,y^{(1)} = A*x^{(1)} x^{(1)} = \frac{y^{(1)}}{\mu_1}, \mu_1 = ||x^{(1)}||$
	
	
	\subsection{Построение чебышевской сетки на выбранном интервале:}
	
	Строим чебышевскую сетку на отрезке [a, b] на k узлах. Сетку записываем в массив $x^h$:
	$t_k \in [-1, 1], $
	$x^h \in [a, b]$
	
	$t_k = 0 $
	
	для k от 0 до n:
	
	\quad $t_k = \cos(\frac{\pi \cdot (2k+1)}{2 \cdot (n+1)})$
	
	\quad $x^h_k = \frac{a+b}{2} + \frac{b-a}{2} \cdot t_k$
	
	конец цикла.
	
	\subsection{Условия применимости метода}
	Критерии существования и единственности интерполяционного полинома: 
	
	1. Степень полинома должна быть на 1 меньше, чем количество точек.
	
	2. $x_i$ должны быть попарно различны
	
	Проверка: 
	
	1.  Табличная функция задана: $(x_i,y_i ),i=0,…,n$. Следовательно, количество точек $(n+1)$. Полином Лагранжа строится по формуле $L_n (x)=\sum_{i=0}^{n}y_i\cdot\prod\limits_{k = 0, k \neq i}^n \frac{(x - x_k)}{(x_i-x_k)}$, степень этого полинома n. Поэтому степень полинома Лагранжа всегда на 1 меньше, чем количество точек. Условие выполнено.
	
	2. Мы строим сетку с учетом того, что $x_i$ не повторяются. Потому что в случае равномерной сетки мы к одному и тому же числу прибавляем разные числа, которые не повторяются. А в случае чебышевской сетки мы сначала считаем корни полинома Чебышева на отрезке $[0, 1]$, они различны. Затем переводим их в наш отрезок интерполяции, что тоже приводит к различным узлам сетки. 
	
	\section{Предварительный анализ задачи}
	Согласно теореме: если матрица вещественная и симметричная, то она является матрицей простой структуры. Будем задавать матрицы, удовлетворяющие данному критерию.
	
	Такие матрицы будем искать с помощью QR-разложения из пакета Matlab, где матрица Q – ортогональная, R - верхнетреугольная.(Генерируются матрица, полученную матрицу представляем в виде QR-разложения, т.е. умножения 2 матриц Q и R). С помощью QR-разложения находим матрицу Q.
	
	Задается диагональная матрица D, у нее на диагонали – собственные числа. 
	
	Нужная симметричная матрица получается следующим образом:
	$A = Q \cdot D \cdot Q^T$
	
	\section{Тестовый пример с детальными расчётами для задачи малой размерности}
	
	\section{Подготовка контрольных тестов}
	\paragraph{}Создаётся симметричная матрица  $A_{10 \times 10}$  для нахождения собственных чисел с помощью метода скалярных произведений с точностью  $\epsilon=10^{-i}$, где $i \in [0, 14], i \in \mathbb {N} $. 
	Собственные числа:[-10, 10, 1, 8, 2, 7, 6, 5, 3, 4]
	
	\section{Модульная структура программы}
	ScalarProductMethod(вх: $A, \epsilon$, вых: $\lambda_{1,2}, w_{1,2}, n$) Находит  максимальные собственные числа разные по знаку матрицы А  $\lambda_{1,2}$ и соответствующие им собственные вектора $w_{1,2}$ с помощью метода скалярных произведений за n итераций с заданной точностью $\epsilon$.
	
	
	\section{Анализ результатов}
	1.По графику зависимости количества итераций метода скалярных произведений от заданной точности видно, что количество итераций увеличивается линейно [4 5 7 9 11 14 16 19 24 29 34 39 44 50].
	При изменении $\epsilon$ на порядок количество итераций изменяется примерно на 20. 
	
	2. Графики заивисимости с.в. и невязок от заданной точности линейны.
	
	
	\section{Выводы}
	\begin{enumerate}
	\item Увеличение количества узлов сетки, на которой строится интерполяционный полином, не приводит к снижению погрешности. 
	\item При выборе большого числа узлов в случае равномерной сеток норма вектора ошибки интерполяции возрастает. Для сетки Чебышева при большом числе узлов ошибка будет увеличиваться, но не так сильно, как в случае с равномерной сеткой. Это значит, что выбор большого числа точек только ухудшает процесс интерполяции. 
	\item Для построения интерполяционного полинома лучше использовать сетку Чебышева. Такая сетка в большинстве случаев уменьшит ошибку построения интерполяционного полинома.
	\end{enumerate}
	
\end{document}