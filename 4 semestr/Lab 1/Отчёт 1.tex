\documentclass{article}
\usepackage[T2A]{fontenc}
\usepackage[utf8]{inputenc}
\usepackage[russian]{babel}
\usepackage{alltt}
\usepackage{amsmath}
\usepackage{amsfonts}
\usepackage{indentfirst}
\usepackage{layout}
\usepackage{geometry}
\geometry{
	a4paper,
	top=25mm, 
	right=15mm, 
	bottom=25mm, 
	left=30mm
}

\title{non-linear equations}
\author{Иван Золин}
\date{April 2022}
\thispagestyle{empty}
\begin{document}
	
	\large
	\begin{center}
		
		Санкт-Петербургский политехнический университет\\
		Высшая школа прикладной математики и вычислительной физики, ФизМех
		
		~\\
		~\\
		~\\
		~\\
		Направление подготовки\\
		«01.03.02 Прикладная математика и информатика»
		
		~\\
		~\\
		~\\
		~\\
		Отчет по лабораторной работе № 1\\
		\textbf{тема "Интерполяция табличных функций" }	
		~\\	Дисциплина "Численные методы"
	\end{center}
	
	~\\
	~\\
	~\\
	~\\
	~\\
	~\\
	~\\
	\begin{alltt}
		\begin{center}
			Выполнил студент гр. 5030102/00001			  		   		  Золин И.М.
			Преподаватель: 				              	        	Добрецова С.Б.
		\end{center}
		
	\end{alltt}
	
	~\\
	~\\
	~\\
	~\\
	~\\
	~\\
	~\\
	~\\
	~\\
	~\\
	~\\
	~\\
	~\\
	~\\
	~\\
	~\\
	\begin{center}
		Санкт-Петербург
		
		~\\
		\textbf{2022}
		
	\end{center}{}
	
	\newpage
	
	\section{Формулировка задачи и её формализация}
	\subsection{Формулировка задания}
	Нам дан набор точек $(x_i,y_i), i = 0,..., n$ и их количество (n+1).
	
	пусть $x^h = \left\{x_i\right\}^n_{i=0}$ - сетка, $y^h = \left\{y_i\right\}^n_{i=0}$ - сеточная функция. Пусть табличная функция задана парой элементов $(x^h,y^h)$. Требуется построить функцию $\phi(x)$ в форме интерполяционного полинома Лангранжа, которая удовлетворяет критерию близости:
	
	$\phi(x) \approx (x^h,y^h)$
	
	Для построения полинома необходимо использовать чебышевскую сетку.
	
	Также нужно исследовать влияние количества узлов на точность интерполяции и сходимость интерполяционного процесса.
	
	\subsection{Формулировка задания}
	1. Для малого числа узлов (3..10) вычислить значения полинома Лагранжа на чебышевской сетке и фактической ошибки – разность между значением функции и полинома
	
	2. Построить графики функции и 3х полиномов для различного числа узлов (n=3..10), фактической ошибки для тех же 3х полиномов
	
	3. К линиям фактической ошибки добавить линию для теоретической ошибки, построенной для одного из 3х полиномов
	
	4. Построить график максимальной ошибки на отрезке в зависимости от числа узлов
	
	\section{Алгоритм метода и условия его применимости}   
	
	\subsection{Построение интерполяционного полинома в форме Лангранжа:}
	\subsubsection{Алгоритм}
	1. Входные данные: пусть х – аргумент интерполяционного полинома; $x^h,y^h$ – сетка и сеточная функция;  
	
	2. Цикл:
	
	k = 0, i = 0, $basics = 0, res = 0$
	
	для i от 0 до size: 
	
	\quad mul = 1.0;
	
	\quad для k от 0 до size:
		
	\qquad если k не равно i:
				
	\quad \qquad $basics *=\frac{(x - x^h_k)}{(x^h_i-x^h_k)}$;
			
	\qquad конец «если»
		
	\quad конец цикла
		
	\quad res += $y^h_i * basics$
	
	конец цикла
	
	вернуть res;

	
	Конец
	
	3. Результат: $res$
	~\\
	
	Построение чебышевской сетки на выбранном интервале:
	
	$x^{(0)} \in R^n,y^{(1)} = A*x^{(1)} x^{(1)} = \frac{y^{(1)}}{\mu_1}, \mu_1 = ||x^{(1)}||$
	
	
	\subsection{Построение чебышевской сетки на выбранном интервале:}
	
	Строим чебышевскую сетку на отрезке [a, b] на k узлах. Сетку записываем в массив $x^h$:
	$t_k \in [-1, 1], $
	$x^h \in [a, b]$
	
	$t_k = 0 $
	
	для k от 0 до n:
	
	\quad $t_k = \cos(\frac{\pi \cdot (2k+1)}{2 \cdot (n+1)})$
	
	\quad $x^h_k = \frac{a+b}{2} + \frac{b-a}{2} \cdot t_k$
	
	конец цикла.
	
	\subsection{Условия применимости метода}
	Критерии существования и единственности интерполяционного полинома: 
	
	1. Степень полинома должна быть на 1 меньше, чем количество точек.
	
	2. $x_i$ должны быть попарно различны
	
	Проверка: 
	
	1.  Табличная функция задана: $(x_i,y_i ),i=0,…,n$. Следовательно, количество точек $(n+1)$. Полином Лагранжа строится по формуле $L_n (x)=\sum_{i=0}^{n}y_i\cdot\prod\limits_{k = 0, k \neq i}^n \frac{(x - x_k)}{(x_i-x_k)}$, степень этого полинома n. Поэтому степень полинома Лагранжа всегда на 1 меньше, чем количество точек. Условие выполнено.
	
	2. Мы строим сетку с учетом того, что $x_i$ не повторяются. Потому что в случае равномерной сетки мы к одному и тому же числу прибавляем разные числа, которые не повторяются. А в случае чебышевской сетки мы сначала считаем корни полинома Чебышева на отрезке $[0.5, 1.5]$, они различны. Затем переводим их в наш отрезок интерполяции, что тоже приводит к различным узлам сетки. 
	
	\section{Предварительный анализ задачи}
	Создаём чебышевскую сетку на отрезке $[0.1, 1.5]$, т.к. на этом
	отрезке функция $f(x) = lg(x) + \frac{7}{2x+6}$ является непрерывной

	
	\section{Тестовый пример с детальными расчётами для задачи малой размерности}
	Рассмотрим функцию $y = \sqrt{2x+ln(x)} , x \in [1, 5]$. 
	
	Возьмём 4 узла : $x_0=1, x_1=2,x_2=e,x_3=5, y_0 = f(x_0) = 1, y_1 = f(x_1) = 2.074, y_2 = f(x_2) = 2.537, y_3 = f(x_3) = 3.271$
	
	$L_3 (x)=\sum_{i=0}^{3}y_i\cdot\prod\limits_{k = 0, k \neq i}^3 \frac{(x - x_k)}{(x_i-x_k)} = 1\cdot\frac{(x-2)(x-e)(x-5)}{(1-2)(1-e)(1-5)} +2.074\cdot\frac{(x-1)(x-e)(x-5)}{(2-1)(2-e)(2-5)}  +2.537\cdot\frac{(x-1)(x-2)(x-5)}{(e-1)(e-2)(e-5)}+3.271\cdot\frac{(x-1)(x-2)(x-e)}{(5-1)(e-2)(5-e)} \approx \frac{-1}{6.873} (x^3-9.71x^2+28.97x-2.71) + \frac{2.074}{2.155} (x^3-8.71x^2+21.26x-13.55) - \frac{2.537}{2.816} (x^3-8x^2+17x-10) + \frac{3.271}{27.381} (x^3-5.71x^2+10.13x-5.42) \approx 0.039x^3-0.501x^2+2.245x -0.803$
	
	Ошибки вычислений: $R_n(x) = y - L_n(x)$
	\begin{center}
		\begin{tabular}{ |c|c|c|c|c|c|c|} 
			\hline
			x & 1 & 1.5 & 2 & e & 3 & 5  \\ 
			$R_n(x)$ & 0.43 & 0.31 & 0.17 & 0.15 & 0.188 & 0.499  \\ 
			\hline
		\end{tabular}
	\end{center}
	\section{Подготовка контрольных тестов}
	\paragraph{}Находим значения функции $y_i = f(x_i), i = 0, .., n,$ где n - количсетво узлов. Количество узлов меняется от 3 до 7 для построения полинома и вычисления фактической ошибки и от 3 до 30 для вычисления макимальной ошибки
	
	\section{Модульная структура программы}
	ChebyshevGrid(вх: $a, b, n,$ вых: $x_i, y_i$). Строит сетку Чебышева размера n на заданном отрезке $[a, b]$
	
	LagrangePolynom(вх: $x^h, y^h, x, n,$ вых: $P(x)$). Находит значения полинома Лагранжа в точках вектора x, где n-количество узлов, а $x^h$,- значения Чебышевской сетки, $y^h$ - - значения Чебышевской сеточной функции
	
	\section{Анализ результатов}
	\begin{enumerate}
		\item Разрыв графика ошибки для 3, 5, 7 узлов от значений х показывает, что значения полинома и исходной функции совпали
		\item На графике зависимости максимальной ошибки от числа узлов наблюдается достижение машинной точности при 25 узлах.
		\item При увелечении числа узлов наблюдается улучшение фактичсекой ошибки
	\end{enumerate}
	\section{Выводы}
	\begin{enumerate}
	\item Увеличение количества узлов сетки, на которой строится интерполяционный полином, не приводит к снижению погрешности. 
	\item Этот метод требует достаточно большое количество вычислений, но при этом для него не нужно считать производные 
	\item При этом для опредленного числа узлов мы можем добиться машинной точности значений полинома
	\end{enumerate}
	
\end{document}