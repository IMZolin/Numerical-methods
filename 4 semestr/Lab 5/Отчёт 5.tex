\documentclass{article}
\usepackage[T2A]{fontenc}
\usepackage[utf8]{inputenc}
\usepackage[russian]{babel}
\usepackage{alltt}
\usepackage{amsmath}
\usepackage{amsfonts}
\usepackage{indentfirst}
\usepackage{layout}
\usepackage{geometry}
\geometry{
	a4paper,
	top=25mm, 
	right=15mm, 
	bottom=25mm, 
	left=30mm
}

\title{non-linear equations}
\author{Иван Золин}
\date{April 2022}
\thispagestyle{empty}
\begin{document}
	
	\large
	\begin{center}
		
		Санкт-Петербургский политехнический университет\\
		Высшая школа прикладной математики и вычислительной физики, ФизМех
		
		~\\
		~\\
		~\\
		~\\
		Направление подготовки\\
		«01.03.02 Прикладная математика и информатика»
		
		~\\
		~\\
		~\\
		~\\
		Отчет по лабораторной работе № 5\\
		\textbf{тема "Решение задачи Коши для ОДУ 1 порядка методами Рунге-Кутты" }	
		~\\	Дисциплина "Численные методы"
	\end{center}
	
	~\\
	~\\
	~\\
	~\\
	~\\
	~\\
	~\\
	\begin{alltt}
		\begin{center}
			Выполнил студент гр. 5030102/00001			  		   		  Золин И.М.
			Преподаватель: 				              	        	Добрецова С.Б.
		\end{center}
		
	\end{alltt}
	
	~\\
	~\\
	~\\
	~\\
	~\\
	~\\
	~\\
	~\\
	~\\
	~\\
	~\\
	~\\
	~\\
	~\\
	~\\
	~\\
	\begin{center}
		Санкт-Петербург
		
		~\\
		\textbf{2022}
		
	\end{center}{}
	
	\newpage
	
	\section{Формулировка задачи и её формализация}
	\subsection{Формулировка задания}
	\begin{enumerate}
		\item Найти численное решение задачи Коши на равномерно сетке модифицированным методом Рунге-Кутта 3 порядка с коэффициентом $\frac{1}{3}$
		\item Получить решения для двух значений шага и построить
		графики точного и полученных решений на отрезке, график
		ошибки на заданном отрезке
		\item Построить графики зависимости фактической точности и
		числа итераций от заданной точности (заданная точность
		достигается по правилу Рунге);
		\item Внести в начальное условие возмущение и построить коэффициентом график зависимости нормы ошибки от величины возмущения при
		фиксированной точности.

	\end{enumerate}	

	\subsection{Формализация задания}
	Дано дифференциальное уравнение 1-го порядка $F(x, y, y') = 0$,
	где $y(x)$ - неизвестная функция. Поставлена задача Коши
	\begin{equation*}
		\begin{cases}
			y' = f(x,y), x \in [a, b]\\
			y(a) = y_0
		\end{cases}
	\end{equation*}
	Необходимо найти приближенное решение этой задачи
	
	\section{Алгоритм метода и условия его применимости}   
	
	\subsection{Алгоритм}
	Входные данные: $a, b,  \varepsilon, f(x,y), t_0,q_0 = y_0$;  

	\begin{enumerate}
		\item Составим равномерную сетку $x_i = a + \frac{b-a}{n}\cdot i$, где $i = \overline{0,n}$
		\item $h = \frac{b-a}{n} $
		\item $t_{i+1} = t_i + \frac{h}{4} \cdot (f(x_i,t_i)+3f(x_{i+\frac{2}{3}},t_i + \frac{2h}{3} f(x_{i+\frac{1}{3}}, t_i+\frac{h}{3}f(x_i,t_i))))$
		\item $h =  \frac{h}{2}$
		\item $q_{i+1} = t_{i}$
		\item $t_{i+1} = t_i + \frac{h}{4} \cdot (f(x_i,t_i)+3f(x_{i+\frac{2}{3}},t_i + \frac{2h}{3} f(x_{i+\frac{1}{3}}, t_i+\frac{h}{3}f(x_i,t_i))))$
		\item Если $\frac{|q_{i+1} - t_{i+1}|}{2^p-1} \geq \varepsilon$, где p = 3 - порядок, то возвращаемся к пункту 4, иначе возвраащем $t_{i+1}$.
	\end{enumerate}
	
	Результат: $t_{i+1}$
	
	\subsection{Условия применимости метода}
	Для того, чтобы алгоритм находил решения с нужной точностью, нужно чтобы выполнялись условия теоремы о существовании и единственности задачи Коши, то есть функции $f(x, y)$ и $f'_y(x, y)$ должны быть непрерывными на $[a, b]$;
	непрерывными на $[a, b]$;

	\section{Предварительный анализ задачи}
	Создаётся равномерная сетка на отрезке $[a, b]$.
	
	\section{Тестовый пример с детальными расчётами для задачи малой размерности}
	Решим задачу Коши $y(a + 0.2)$ для функции $y' = \frac{y}{x} + x\cos(x)$ на отрезке $[\frac{\pi}{2}, 2\pi] ,
	 y(\frac{\pi}{2})=\frac{\pi}{2}, p = 3$
	\begin{enumerate}
		\item h = 0.2
		
		$k_1 = f(x,y) = f(\frac{\pi}{2}, \frac{\pi}{2})= 1\\$
		$k_2 = f(x + \frac{h}{3},y + \frac{hk_1}{3})= f(x + \frac{h}{3},y + \frac{h}{3}f(x,y))= f(x + \frac{h}{3},y + \frac{h}{3}) = f(\frac{\pi}{2} + \frac{0.2}{3},\frac{\pi}{2} + \frac{0.2}{3})= 1 + (\frac{\pi}{2} + \frac{0.2}{3})\cos(\frac{\pi}{2} + \frac{0.2}{3}) = 1 - 1.637463\cdot0.066617 = 0.890917\\$
		$k_3 = f(x + \frac{2h}{3},y + \frac{2hk_2}{3})= f(x + \frac{2h}{3},y + \frac{2h}{3}f(x + \frac{h}{3},y + \frac{h}{3})) = f(\frac{\pi}{2} + \frac{0.4}{3},\frac{\pi}{2} + \frac{0.4 }{3}0.890917)= \frac{\frac{\pi}{2}+\frac{0.4}{3}\cdot0.890917}{\frac{\pi}{2}+\frac{0.4}{3}} + (\frac{\pi}{2}+\frac{0.4}{3})\cos{(\frac{\pi}{2}+\frac{0.4}{3})} = 0.991465-1.704130\cdot0.132939 = 0.764920\\$
		$y_1 = y + \frac{h(k_1+3k_3)}{4} = y + \frac{h}{4}(f(x,y)+3f(x + \frac{2h}{3},y + \frac{2h}{3}f(x + \frac{h}{3},y + \frac{h}{3}))) =\frac{\pi}{2} + \frac{0.2}{4}(1+2.294761) = \frac{\pi}{2} + 0.164738 =1.735534 \\$
		\item h = 0.1
		
		$k_1 = f(x,y) = f(\frac{\pi}{2}, \frac{\pi}{2})= 1\\$
		$k_2 = f(x + \frac{h}{3},y + \frac{hk_1}{3})= f(x + \frac{h}{3},y + \frac{h}{3}f(x,y))= f(x + \frac{h}{3},y + \frac{h}{3}) = f(\frac{\pi}{2} + \frac{0.1}{3},\frac{\pi}{2} + \frac{0.1}{3})= 1 + (\frac{\pi}{2} + \frac{0.1}{3})\cos(\frac{\pi}{2} + \frac{0.1}{3}) = 1 - 1.604130\cdot0.033327 = 0.946539\\$
		$k_3 = f(x + \frac{2h}{3},y + \frac{2hk_2}{3})= f(x + \frac{2h}{3},y + \frac{2h}{3}f(x + \frac{h}{3},y + \frac{h}{3})) = f(\frac{\pi}{2} + \frac{0.2}{3},\frac{\pi}{2} + \frac{0.2 }{3}0.946539)= \frac{\frac{\pi}{2}+\frac{0.2}{3}\cdot0.946539}{\frac{\pi}{2}+\frac{0.2}{3}} + (\frac{\pi}{2}+\frac{0.2}{3})\cos{(\frac{\pi}{2}+\frac{0.2}{3})} = 0.997823-1.637463\cdot0.066617 = 0.888740\\$
		$y_1 = y + \frac{h(k_1+3k_3)}{4} = y + \frac{h}{4}(f(x,y)+3f(x + \frac{2h}{3},y + \frac{2h}{3}f(x + \frac{h}{3},y + \frac{h}{3}))) = \frac{\pi}{2} + 0,164708 = 1,735504 \\$
	\end{enumerate}

	Правило Рунге: $\frac{y_2- y_1}{2^p-1} = \frac{1.735534 - 1,735504}{7} = 0.00000424$
	
	Точное значение: $y(a + 0.2) = y(1.7707) = 1,735498$
	\section{Подготовка контрольных тестов}
	\paragraph{}Строится равномерная сетка на отрезке $[\frac{\pi}{2}, 2\pi]$ для решения задачи Коши методом Рунге-Кутта 3 порядка с точностью $\varepsilon = 10^{-3}, ..., 10^{-8}$ для функции $y' = \frac{y}{x} + x\cos(x)$ на отрезке $y(\frac{\pi}{2}) = \frac{\pi}{2}$.

	\section{Модульная структура программы}
	RungeKuttaMethod(вх: h, m, x, y;вых: res) Находит значение res в следующей точке x + h при m разбиений при значении y в точке x
	
	Cycle(вх: x, y, $\varepsilon$, вых: res). Находит значение res в следующей точке x + h при m разбиений при значении y в точке x, удовлетворяющее условию, где $\frac{y^h_{i+1} - y^{h/2}_{i+1}}{2^p-1} < \varepsilon $, где $p=3$
	\section{Анализ результатов}
	\begin{enumerate}
		\item Из графика зависимости фактической ошибки от заданной точности можно заметить, что желаемая точность достигается. 
		\item График зависимости максимальной ошибки от заданной точности показывает, что не достигается заданная точность, но
		при этом не превосходит $\varepsilon \cdot N$
		\item График зависимости числа итераций от заданной точности
		показывает, что с увеличением точности число итераций
		увеличичвается линейно
		\item График зависимости максимальной ошибки от возмущения
		начального показывает, что при достижении величины возмущения заданной точности макисмальная ошибка перестаёт падать
	\end{enumerate}
	
	\section{Выводы}
	\begin{enumerate}
		\item Метод Рунге-Кутты, относящийся к группе методов 3-го порядка, прост в реализации и гарантированно может найти решение с заданной точностью.
		\item С уменьшением шага уменьшается ошибка в вычислениях
		\item  При ошибке начального значения меньше заданной точности ошибка не влияет на результат

	\end{enumerate}
	
\end{document}