\documentclass{article}
\usepackage[T2A]{fontenc}
\usepackage[utf8]{inputenc}
\usepackage[russian]{babel}
\usepackage{alltt}
\usepackage{amsmath}
\usepackage{amsfonts}
\usepackage{indentfirst}
\usepackage{layout}
\usepackage{geometry}
\geometry{
	a4paper,
	top=25mm, 
	right=15mm, 
	bottom=25mm, 
	left=30mm
}

\title{non-linear equations}
\author{Иван Золин}
\date{April 2022}
\thispagestyle{empty}
\begin{document}
	
	\large
	\begin{center}
		
		Санкт-Петербургский политехнический университет\\
		Высшая школа прикладной математики и вычислительной физики, ФизМех
		
		~\\
		~\\
		~\\
		~\\
		Направление подготовки\\
		«01.03.02 Прикладная математика и информатика»
		
		~\\
		~\\
		~\\
		~\\
		Отчет по лабораторной работе № 5\\
		\textbf{тема "Решение задачи Коши для ОДУ 1 порядка методами Рунге-Кутты" }	
		~\\	Дисциплина "Численные методы"
	\end{center}
	
	~\\
	~\\
	~\\
	~\\
	~\\
	~\\
	~\\
	\begin{alltt}
		\begin{center}
			Выполнил студент гр. 5030102/00001			  		   		  Золин И.М.
			Преподаватель: 				              	        	Добрецова С.Б.
		\end{center}
		
	\end{alltt}
	
	~\\
	~\\
	~\\
	~\\
	~\\
	~\\
	~\\
	~\\
	~\\
	~\\
	~\\
	~\\
	~\\
	~\\
	~\\
	~\\
	\begin{center}
		Санкт-Петербург
		
		~\\
		\textbf{2022}
		
	\end{center}{}
	
	\newpage
	
	\section{Формулировка задачи и её формализация}
	\subsection{Формулировка задания}
	Пусть дано обыкновенное дифференциальное уравнение (ОДУ) первого порядка:
	\begin{center}
	$y' = f(x,y); x_0 \leq x \leq b$
	\end{center}
	Будем считать, что для данной задачи Коши выполняются все требования, обеспечивающие существование и единственность на отрезке $[x_0; b]$ ее решение $y = y(x)$.
	
	Решить такое уравнение, то есть найти общее решение у = y(x,C) с тем, чтобы выделить из него интегральную кривую у = у(x), проходящую через заданную точку $(x_0, y_0)$ удается лишь для некоторых специальных типов таких уравнений. В остальных случаях необходимо пользоваться приближенными
	способами решения задач.
	
	В данной лабораторной работе необходимо решить следующую задачу Коши методом Рунге-Кутты 3 порядка:
	
	Метод Рунге-Кутты 3-го порядка относится к явным методам. Явные методы являются наиболее простыми в реализации: вспомогательные значения $K_i$ вычисляются последовательно, каждое последующее из них явно выражается через уже найденные.
	Построим формулу следующего вида:

	\subsection{Формализация задания}
	Требуется найти значение интеграла Римана $\int_a^b f(x)dx$ функции $f(x) = x^5 - 6.2x^3 + 3.5x^2 -7x -2.1$ на отрезке [-2.9, 0.4] с помощью формулы Симпсона исследовать:
	
	- зависимость погрешности от измельчения шага;
	
	- сравнение теоретической и фактической погрешностей;
	
	- влияние заданной точности на количество вычислений.
	\section{Алгоритм метода и условия его применимости}   
	
	\subsection{Алгоритм}
	Входные данные: $a, b,  eps, f(x)$;  
	
	$I_{prev} = 0, m = 1$
	\begin{enumerate}
		\item $S_1 = 0, S_2 = 0$
		\item Вычисляем шаг $h$ по формуле: $h = \frac{b-a}{2m} $
		\item $S_1 = S_1 + f(a + ih)$,для $i$ от 1 до $2m-1$ с шагом 2
		\item $S_2 = S_2 + f(a + ih)$,для $i$ от 1 до $2m-2$ с шагом 2
		\item $I = \frac{h}{3}(f(a)+f(b)+4S_1+2S_2)$
		\item $I_{prev} = I$
		\item $m = m * 2$
		\item Выполнить шаги 1-5
		\item Если $\frac{|I-I_{prev}|}{15} \geq \epsilon$ то возвращаемся к пункту 6, иначе возвраащем $I$.
	\end{enumerate}
	
	Результат: $I$
	
	\subsection{Условия применимости метода}
	Для того, чтобы алгоритм находил решения с нужной точностью, нужно чтобы выполнялись условия теоремы о существовании и единственности задачи Коши, то есть функции $f(x, y)$ и $f'_y(x, y)$ должны быть непрерывными на $[x_0, b]$;
	рерывными на [x0; b];
	Так как нам дано ОДУ 1 порядка, то сделаем замену y’=z, получим:

	
	\section{Предварительный анализ задачи}
	Для функции $f(x) = x^5 - 6.2x^3 + 3.5x^2 -7x -2.1$ на отрезке
	вычиcляем величину шага h в заивисмости от n.
	
	\section{Тестовый пример с детальными расчётами для задачи малой размерности}
	Пусть дано ОДУ
	
	\begin{enumerate}
		\item $h = \frac{\pi - 0}{2}\\
		I_1 = \frac{h}{3}(f(0)+f(\pi)+4f(\frac{\pi}{2})) = 6.57974\\
		error = |6.28318 - 6.57974| = 0.29656$ \\
		\item $h = \frac{\pi - 0}{4}\\
		I_2 = \frac{h}{3}(f(0)+f(\pi)+4(f(\frac{\pi}{4}) + f(\frac{3\pi}{4}))+ 2f(\frac{\pi}{2})) = 6.29751\\
		error = |6.28318 - 6.29751| =  0.01433\\$ 
		Ошибка по правилу Рунге:
		%$runge \frac{|I_1-|}{den} = 0.46324$
		$runge = \frac{|I_1- I_2|}{15} = 0,01881$
		\item $h = \frac{\pi - 0}{8}\\
		I_3 = \frac{h}{3}(f(0)+f(\pi)+4(f(\frac{\pi}{8}) + f(\frac{3\pi}{8}) + f(\frac{5\pi}{8}) + f(\frac{7\pi}{8}))+ 2(f(\frac{\pi}{4})+ f(\frac{\pi}{2})) + f(\frac{\pi}{2})) = 6.57974\\
		error = |6.28318 - 6.57974| =  0.00085\\$
		Ошибка по правилу Рунге:
		$runge = \frac{|I_2- I_3|}{15} = 0,00089$
	\end{enumerate}
	
	\section{Подготовка контрольных тестов}
	\paragraph{}Находим значения функции f(xi) и значения интеграла по
	формуле Ньютона-Лейбница $fIntegr(x_i), i = 0, .., n,$ где n - количсетво узлов. Количество узлов меняется, начиная с 1, и
	с каждым шагом умножается на 2.
	
	\section{Модульная структура программы}
	func(вх: x, вых: $f$). Находит значение функции в точке x
	
	funcIntegr(вх: x, вых: $fIntegr$). Находит интеграл $fIntegr$
	по формуле Ньютона-Лейбница в точке x.
	
	
	SimpsonMethod(вх: $a, b, f, \epsilon$, вых: $I, iter$) - считает интеграл с помощью метода Симпсона. Входные аргументы: $[a,b]$ - отрезок, на котором считается интеграл, $f$ - интегрируемая функция, $\epsilon$ - заданная точность. Возвращаемое значение: $iter$ - количество разбиений при котором достигается заданная точность, $I$ - вычисленный интеграл с заданной точностью $\epsilon$.
	
	\section{Анализ результатов}
	Для начала построим графики, показывающие достигается ли точность вычислений по правилу Рунге.
	Это будет график зависимости фактической ошибки вычислений от точности.
	По оси абсцисс отложим точность, по оси ординат – абсолютную погрешность.
	Для наглядности изобразим на графике биссектрису первой четверти. Если график будет лежать ниже
	ее, это будет значить, что точность вычислений достигается.
	
	\begin{enumerate}
		\item Из графика зависимости фактической ошибки от заданной точности можно заметить, что желаемая точность достигается. 
		\item График зависимости числа итераций от заданной точности показывает, что при улучшении точности число итераций увеличивается
		\item Из графика зависимости числа итераций от заданной точности можно заметить, что после заданной точности $\epsilon = 10^-6$ количество итераций не меняются, как и фактическая ошибка для 1-ого графика.
		\item График зависимости точности от длины разбиений подтверждает, что порядок точности применяемой формулы равен 4, а константа равна 2.8.
	\end{enumerate}
	
	\section{Выводы}
	\begin{enumerate}
		\item Метод Рунге-Кутты, относящийся к группе методов 3-го порядка, прост в реализации и гарантированно может найти решение с заданной точностью.
		\item С уменьшением шага интегрирования уменьшается абсолютная погрешность.
		\item При увеличении погрешности в начальном условии задачи Коши норма вектора ошибки будет также увеличиваться. При этом чем выше погрешность, тем выше может быть ошибка.

	\end{enumerate}
	
\end{document}