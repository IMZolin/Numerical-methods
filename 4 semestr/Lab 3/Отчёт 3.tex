\documentclass{article}
\usepackage[T2A]{fontenc}
\usepackage[utf8]{inputenc}
\usepackage[russian]{babel}
\usepackage{alltt}
\usepackage{amsmath}
\usepackage{amsfonts}
\usepackage{indentfirst}
\usepackage{layout}
\usepackage{geometry}
\geometry{
	a4paper,
	top=25mm, 
	right=15mm, 
	bottom=25mm, 
	left=30mm
}

\title{non-linear equations}
\author{Иван Золин}
\date{April 2022}
\thispagestyle{empty}
\begin{document}
	
	\large
	\begin{center}
		
		Санкт-Петербургский политехнический университет\\
		Высшая школа прикладной математики и вычислительной физики, ФизМех
		
		~\\
		~\\
		~\\
		~\\
		Направление подготовки\\
		«01.03.02 Прикладная математика и информатика»
		
		~\\
		~\\
		~\\
		~\\
		Отчет по лабораторной работе № 3\\
		\textbf{тема "Решение интегралов с помощью квадратурных формул Ньютона-Котеса" }	
		~\\	Дисциплина "Численные методы"
	\end{center}
	
	~\\
	~\\
	~\\
	~\\
	~\\
	~\\
	~\\
	\begin{alltt}
		\begin{center}
			Выполнил студент гр. 5030102/00001			  		   		  Золин И.М.
			Преподаватель: 				              	        	Добрецова С.Б.
		\end{center}
		
	\end{alltt}
	
	~\\
	~\\
	~\\
	~\\
	~\\
	~\\
	~\\
	~\\
	~\\
	~\\
	~\\
	~\\
	~\\
	~\\
	~\\
	~\\
	\begin{center}
		Санкт-Петербург
		
		~\\
		\textbf{2022}

	\end{center}{}
	
	\newpage
	
	\section{Формулировка задачи и её формализация}
	\subsection{Формулировка задания}
	Пусть требуется найти значение интеграла Римана $\int_a^b f(x)dx$ которой заданной на отрезке $[a, b]$ функции $f(x)$. Известно, что для функций, допускающих на этом промежутке конечное число точек разрыва первого рода, такое значение существует, единственно и может быть формально получено по
	определению:

	$I =  \lim_{n\to\infty} \sum_{i=1}^{n} f(\xi_i) (x_i-x_{i-1}) $
	
	где $[x_i]^n_{i=0}$ произвольная упорядоченная система точек отрезка $[a,b]$ такая, что $max[x_0-a;x_i-x_{i-1};b-x_n] \to 0$ при $n \to \infty$, а $\xi_i$, такая точка $x_{i-1} \leq \xi_i \leq x_i$.
	В математическом анализе обосновывается аналитический способ нахождения значения определенного
	интеграла с помощью формулы Ньютона-Лейбница $I = F(b)-F(a)$, где $F(x)$ есть некоторая первообразная
	функции $f(x)$.
	Но, не всегда можно воспользоваться этим подходом. Во-первых, первообразная среди элементарных
	функций не существует для большинства элементарных функций. Во-вторых, вычисление $F(b)-F(a)$
	может быть значительно сложнее, чем вычисление существенно большего числа $f(x)$.
	В этой лабораторной работе используется формула Симпсона.
	Разобьем отрезок $[a,b]$ на $2N$ интервалов длиной $h = \frac{b-a}{2N}$.
	
	$x_k = a + kh, k = 0,...,2N$

	Тогда формула правых Симпсона примет вид:
	
	$S_{3,N}(f) = \frac{h}{3}(f(a)+f(b)+4\sum_{k=1}^{N}f(x_{2k-1})+2\sum_{k=1}^{N-1}f(x_{2k}))$
	
	Для оценки погрешности вычисления определенного интеграла воспользуемся правилом Рунге:
	$\Delta_{2n} \approx \frac{|I_n-I_{2n}|}{2^m-1}$
	для формулы Симпсона m = 4
	
	\subsection{Формализация задания}
		Требуется найти значение интеграла Римана $\int_a^b f(x)dx$ функции $f(x) = x^5 - 6.2x^3 + 3.5x^2 -7x -2.1$ на отрезке [-2.9, 0.4] с помощью формулы Симпсона исследовать:
	
	- зависимость погрешности от измельчения шага;
	
	- сравнение теоретической и фактической погрешностей;
	
	- влияние заданной точности на количество вычислений.
	\section{Алгоритм метода и условия его применимости}   

	\subsection{Построение интерполяционного полинома в форме Лангранжа:}
	\subsubsection{Алгоритм}
	Входные данные: $a, b,  eps, f(x)$;  
	
	$I_{prev} = 0, m = 1$
	\begin{enumerate}
		\item $S_1 = 0, S_2 = 0$
		\item Вычисляем шаг $h$ по формуле: $h = \frac{b-a}{2m} $
		\item $S_1 = S_1 + f(a + ih)$,для $i$ от 1 до $2m-1$ с шагом 2
		\item $S_2 = S_2 + f(a + ih)$,для $i$ от 1 до $2m-2$ с шагом 2
		\item $I = \frac{h}{3}(f(a)+f(b)+4S_1+2S_2)$
		\item $I_{prev} = I$
		\item $m = m * 2$
		\item Выполнить шаги 1-5
		\item Если $\frac{|I-I_{prev}|}{15} \geq \epsilon$ то возвращаемся к пункту 6, иначе возвраащем $I$.
	\end{enumerate}
	
	Результат: $I$
	
	\subsubsection{Условия применимости метода}
	Функция $f(x)$ должна быть дважды дифференцируемой на
	отрезке $[a, b]$.
	
	\section{Предварительный анализ задачи}
	Для функции $f(x) = x^5 - 6.2x^3 + 3.5x^2 -7x -2.1$ на отрезке
	 вычиcляем величину шага h в заивисмости от n.
	
	\section{Тестовый пример с детальными расчётами для задачи малой размерности}
	Пусть дана функция $f(x) = 2xsinx$. Требуется найти значение определённого интеграла $I = \int_0^\pi f(x)dx$ 
	
	Точное значение интеграла $I = \int_0^\pi f(x)dx$ можно вычислить по формуле Ньютона-Лейбница. $I_e = 2\pi = 6.28318$
	
	\begin{enumerate}
		\item $h = \frac{\pi - 0}{2}\\
		I_1 = \frac{h}{3}(f(0)+f(\pi)+4f(\frac{\pi}{2})) = 6.57974\\
		error = |6.28318 - 6.57974| = 0.29656$ \\
		\item $h = \frac{\pi - 0}{4}\\
		I_2 = \frac{h}{3}(f(0)+f(\pi)+4(f(\frac{\pi}{4}) + f(\frac{3\pi}{4}))+ 2f(\frac{\pi}{2})) = 6.29751\\
		error = |6.28318 - 6.29751| =  0.01433\\$ 
		Ошибка по правилу Рунге:
		%$runge \frac{|I_1-|}{den} = 0.46324$
		$runge = \frac{|I_1- I_2|}{15} = 0,01881$
		\item $h = \frac{\pi - 0}{8}\\
		I_3 = \frac{h}{3}(f(0)+f(\pi)+4(f(\frac{\pi}{8}) + f(\frac{3\pi}{8}) + f(\frac{5\pi}{8}) + f(\frac{7\pi}{8}))+ 2(f(\frac{\pi}{4})+ f(\frac{\pi}{2})) + f(\frac{\pi}{2})) = 6.57974\\
		error = |6.28318 - 6.57974| =  0.00085\\$
		Ошибка по правилу Рунге:
		$runge = \frac{|I_2- I_3|}{15} = 0,00089$
	\end{enumerate}
	
	\section{Подготовка контрольных тестов}
	\paragraph{}Находим значения функции f(xi) и значения интеграла по
	формуле Ньютона-Лейбница $fIntegr(x_i), i = 0, .., n,$ где n - количсетво узлов. Количество узлов меняется, начиная с 1, и
	с каждым шагом умножается на 2.
	
	\section{Модульная структура программы}
	func(вх: x, вых: $f$). Находит значение функции в точке x
	
	funcIntegr(вх: x, вых: $fIntegr$). Находит интеграл $fIntegr$
	по формуле Ньютона-Лейбница в точке x.

	
	SimpsonMethod(вх: $a, b, f, \epsilon$, вых: $I, iter$) - считает интеграл с помощью метода Симпсона. Входные аргументы: $[a,b]$ - отрезок, на котором считается интеграл, $f$ - интегрируемая функция, $\epsilon$ - заданная точность. Возвращаемое значение: $iter$ - количество разбиений при котором достигается заданная точность, $I$ - вычисленный интеграл с заданной точностью $\epsilon$.
	
	\section{Анализ результатов}
	Для начала построим графики, показывающие достигается ли точность вычислений по правилу Рунге.
	Это будет график зависимости фактической ошибки вычислений от точности.
	По оси абсцисс отложим точность, по оси ординат – абсолютную погрешность.
	Для наглядности изобразим на графике биссектрису первой четверти. Если график будет лежать ниже
	ее, это будет значить, что точность вычислений достигается.

	\begin{enumerate}
		\item Из графика зависимости фактической ошибки от заданной точности можно заметить, что желаемая точность достигается. 
		\item График зависимости числа итераций от заданной точности показывает, что при улучшении точности число итераций увеличивается
		\item Из графика зависимости числа итераций от заданной точности можно заметить, что после заданной точности $\epsilon = 10^-6$ количество итераций не меняются, как и фактическая ошибка для 1-ого графика.
		\item График зависимости точности от длины разбиений подтверждает, что порядок точности применяемой формулы равен 4, а константа равна 2.8.
	\end{enumerate}

	\section{Выводы}
	\begin{enumerate}
		\item С увеличением числа шагов метода Симпсона величина теоретической и практической ошибки уменьшаются.
		\item Для достижения большей точности требуется выполнить большее число операций.
		\item Метод Симпсона достаточно прост в реализации и сможет обеспечить высокую точность вычислений.
	\end{enumerate}

\end{document}