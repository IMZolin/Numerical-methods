\documentclass{article}
\usepackage[T2A]{fontenc}
\usepackage[utf8]{inputenc}
\usepackage[russian]{babel}
\usepackage{alltt}
\usepackage{amsmath}
\usepackage{amsfonts}
\usepackage{indentfirst}

\title{non-linear equations}
\author{Иван Золин}
\date{April 2022}
\thispagestyle{empty}
\begin{document}
	
	\large
	\begin{center}
		
		Санкт-Петербургский политехнический университет\\
		Высшая школа прикладной математики и вычислительной физики, ФизМех
		
		~\\
		~\\
		~\\
		~\\
		Направление подготовки\\
		«01.03.02 Прикладная математика и информатика»
		
		~\\
		~\\
		~\\
		~\\
		Отчет по лабораторной работе № 3\\
		\textbf{тема "Решение интегралов с помощью квадратурных формул Ньютона-Котеса" }	
		~\\	Дисциплина "Численные методы"
	\end{center}
	
	~\\
	~\\
	~\\
	~\\
	~\\
	~\\
	~\\
	\begin{alltt}
		Выполнил студент гр. 5030102/00001			  		   		  Золин И.М.
		Преподаватель: 				              	        	Добрецова С.Б.
	\end{alltt}
	
	~\\
	~\\
	\begin{center}
		Санкт-Петербург
		
		~\\
		\textbf{2022}
	\end{center}{}
	
	\newpage
	
	\section{Формулировка задачи и её формализация}
	\subsection{Формулировка задания}
	Пусть требуется найти значение интеграла Римана $\int_a^b f(x)dx$ которой заданной на отрезке $[a, b]$ функции $f(x)$. Известно, что для функций, допускающих на этом промежутке конечное число точек разрыва первого рода, такое значение существует, единственно и может быть формально получено по
	определению:

	$I =  \lim_{n\to\infty} \sum_{i=1}^{n} f(\xi_i) (x_i-x_{i-1}) $
	
	где $[x_i]^n_{i=0}$ произвольная упорядоченная система точек отрезка $[a,b]$ такая, что $max[x_0-a;x_i-x_{i-1};b-x_n] \to 0$ при $n \to \infty$, а $\xi_i$, такая точка $x_{i-1} \leq \xi_i \leq x_i$.
	В математическом анализе обосновывается аналитический способ нахождения значения определенного
	интеграла с помощью формулы Ньютона-Лейбница $I = F(b)-F(a)$, где $F(x)$ есть некоторая первообразная
	функции $f(x)$.
	Но, не всегда можно воспользоваться этим подходом. Во-первых, первообразная среди элементарных
	функций не существует для большинства элементарных функций. Во-вторых, вычисление $F(b)-F(a)$
	может быть значительно сложнее, чем вычисление существенно большего числа $f(x)$.
	В этой лабораторной работе используется формула Симпсона.
	Разобьем отрезок $[a,b]$ на $2N$ интервалов длиной $h = \frac{b-a}{2N}$.
	
	$x_k = a + kh, k = 0,...,2N$

	Тогда формула правых Симпсона примет вид:
	
	$S_{3,N}(f) = \frac{h}{3}(f(a)+f(b)+4\sum_{k=1}^{N}f(x_{2k-1})+2\sum_{k=1}^{N-1}f(x_{2k}))$
	
	Для оценки погрешности вычисления определенного интеграла воспользуемся правилом Рунге:
	$\Delta_{2n} \approx \frac{|I_n-I_{2n}|}{2^m-1}$
	для формулы Симпсона m = 4
	
	\subsection{Формализация задания}
		Требуется найти значение интеграла Римана $\int_a^b f(x)dx$ функции $f(x) = 2x \cdot \sin(x)$ на отрезке [-3.3, 0.9] с помощью формулы Симпсона исследовать:
	
	- зависимость погрешности от измельчения шага;
	
	- сравнение теоретической и фактической погрешностей;
	
	- влияние заданной точности на количество вычислений.
	\section{Алгоритм метода и условия его применимости}   

	\subsection{Построение интерполяционного полинома в форме Лангранжа:}
	\subsubsection{Алгоритм}
	1. Входные данные: пусть х – аргумент интерполяционного полинома; xk,yk – сетка и сеточная функция;  
	
	2. Цикл:
	
	k = 0, i = 0, $basics = 0, res = 0$
	
	для i от 0 до size: 
	
	\quad mul = 1.0;
	
	\quad для k от 0 до size:
	
	\qquad если k не равно i:
	
	\quad \qquad домножить basics на величину $frac{(x - xk[k])}{(xk[i]-xk[k])}$;
	
	\qquad конец «если»
	
	\quad конец цикла
	
	\quad прибавить к res величину yk[i] * basics
	
	конец цикла
	
	вернуть res;
	
	
	Конец
	
	3. Результат: $res$
	~\\
	
	Построение чебышевской сетки на выбранном интервале:
	
	$x^{(0)} \in R^n,y^{(1)} = A*x^{(1)} x^{(1)} = \frac{y^{(1)}}{\mu_1}, \mu_1 = ||x^{(1)}||$
	
	
	\subsubsection{Условия применимости метода}
	Пусть $A \in \mathbb{R}^{n \times n}$, а $\lambda_1, \lambda_2,...,\lambda_n$ - собственные числа, исследуемой матрицы A, причём $\lambda_1 = -\lambda_2$ и $|\lambda_1| = |\lambda_2| > |\lambda_3| \geq |\lambda_4| \geq ... \geq |\lambda_n|$, матрица А - матрица простой структуры и симметричная. Тогда метод скалярных произведений сходится.
	
	\section{Предварительный анализ задачи}
	Согласно теореме: если матрица вещественная и симметричная, то она является матрицей простой структуры. Будем задавать матрицы, удовлетворяющие данному критерию.
	
	Такие матрицы будем искать с помощью QR-разложения из пакета Matlab, где матрица Q – ортогональная, R - верхнетреугольная.(Генерируются матрица, полученную матрицу представляем в виде QR-разложения, т.е. умножения 2 матриц Q и R). С помощью QR-разложения находим матрицу Q.
	
	Задается диагональная матрица D, у нее на диагонали – собственные числа. 
	
	Нужная симметричная матрица получается следующим образом:
	$A = Q \cdot D \cdot Q^T$
	
	\section{Тестовый пример с детальными расчётами для задачи малой размерности}
	Пусть дана функция $f(x) = 2xsinx$. Требуется найти значение определённого интеграла $I = \int_0^\pi f(x)dx$ 
	
	Точное значение интеграла $I = \int_0^\pi f(x)dx$ можно вычислить по формуле Ньютона-Лейбница. $I_e = 2\pi = 6.28318$
	
	\begin{enumerate}
		\item $h = \frac{\pi - 0}{2}\\
		I_1 = \frac{h}{3}(f(0)+f(\pi)+4f(\frac{\pi}{2})) = 6.57974\\
		error = |6.28318 - 6.57974| = 0.29656$ $runge = 0.46324$
		\item $h = \frac{\pi - 0}{4}\\
		I_2 = \frac{h}{3}(f(0)+f(\pi)+4(f(\frac{\pi}{4}) + f(\frac{3\pi}{4}))+ 2f(\frac{\pi}{2})) = 6.29751\\
		error = |6.28318 - 6.29751| =  0.01433$ $runge = 0,01881$
		\item $h = \frac{\pi - 0}{8}\\
		I_3 = \frac{h}{3}(f(0)+f(\pi)+4(f(\frac{\pi}{8}) + f(\frac{3\pi}{8}) + f(\frac{5\pi}{8}) + f(\frac{7\pi}{8}))+ 2(f(\frac{\pi}{4})+ f(\frac{\pi}{2})) + f(\frac{\pi}{2})) = 6.57974\\
		error = |6.28318 - 6.57974| =  0.00085$ $runge = 0,00089$
	\end{enumerate}
	
	\section{Подготовка контрольных тестов}
	\paragraph{}Создаётся симметричная матрица  $A_{10 \times 10}$  для нахождения собственных чисел с помощью метода скалярных произведений с точностью  $\epsilon=10^{-i}$, где  $i \in [0, 14], i \in \mathbb {N} $. 
	Собственные числа:[-10, 10, 1, 8, 2, 7, 6, 5, 3, 4]
	
	\section{Модульная структура программы}
	SimpsonMethod(вх: $a, b, f, \epsilon$, вых: $I, iter$) - считает интеграл с помощью метода Симпсона. Входные аргументы: $[a,b]$ - отрезок, на котором считается интеграл, $f$ - интегрируемая функция, $\epsilon$ - заданная точность. Возвращаемое значение: $iter$ - количество разбиений при котором достигается
	заданная точность, $I$ - вычисленный интеграл с заданной точностью $\epsilon$.
	
	
	\section{Анализ результатов}
	\begin{enumerate}
		\item 
	\end{enumerate}
	1.По графику зависимости количества итераций метода скалярных произведений от заданной точности видно, что количество итераций увеличивается линейно [4 5 7 9 11 14 16 19 24 29 34 39 44 50].
	При изменении $\epsilon$ на порядок количество итераций изменяется примерно на 20. 
	
	2. Графики заивисимости с.в. и невязок от заданной точности линейны.
	
	
	\section{Выводы}
	\begin{enumerate}
		\item С увеличением числа шагов метода Симпсона величины теоретической и практической ошибок уменьшаются.
		\item Для достижения большей точности требуется выполнить большее число операций.
		\item Метод Симпсона достаточно прост в реализации и сможет обеспечить высокую точность вычислений.
	\end{enumerate}

\end{document}