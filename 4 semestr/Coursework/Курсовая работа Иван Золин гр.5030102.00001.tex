\documentclass{article}
\usepackage[T2A]{fontenc}
\usepackage[utf8]{inputenc}
\usepackage[russian]{babel}
\usepackage{alltt}
\usepackage{amsmath}
\usepackage{amsfonts}
\usepackage{indentfirst}
\usepackage{layout}
\usepackage{geometry}
\usepackage{comment}
\geometry{
	a4paper,
	top=25mm, 
	right=15mm, 
	bottom=25mm, 
	left=30mm
}

\title{non-linear equations}
\author{Иван Золин}
\date{April 2022}
\thispagestyle{empty}
\begin{document}
	
	\large
	\begin{center}
		
		Санкт-Петербургский политехнический университет\\
		Высшая школа прикладной математики и вычислительной физики, ФизМех
		
		~\\
		~\\
		~\\
		~\\
		Направление подготовки\\
		«01.03.02 Прикладная математика и информатика»
		
		~\\
		~\\
		~\\
		~\\
		Отчет по курсовой работе\\
		\textbf{тема "Сравнение решения ОДУ 2ого порядка методом Рунге-Кутты 3 порядка и методом конечных разностей 2 порядка" }	
		~\\	Дисциплина "Численные методы"
	\end{center}
	
	~\\
	~\\
	~\\
	~\\
	~\\
	~\\
	~\\
	\begin{alltt}
		\begin{center}
			Выполнил студент гр. 5030102/00001			  		   		  Золин И.М.
			Преподаватель: 				              	        	Добрецова С.Б.
		\end{center}
		
	\end{alltt}
	
	~\\
	~\\
	~\\
	~\\
	~\\
	~\\
	~\\
	~\\
	~\\
	~\\
	~\\
	~\\
	~\\
	~\\
	~\\
	~\\
	\begin{center}
		Санкт-Петербург
		
		~\\
		\textbf{2022}
		
	\end{center}{}
	
	\newpage
	
	\section{Задание}
	
	Дано ОДУ 2-ого порядка $y'' - \tg{x}\cdot y' + 3y = \sin{x}$ и отрезок $[a;b]$: $a = 0, b = \frac{\pi}{2}$. Найти решение задачи Коши для этого уравнения методом Рунге-Кутты 3 порядка. Найти решение краевой задачи для этого уравнения методом конечных разностей 2 порядка. Провести сравнение результатов по графикам ошибок на отрезке для двух значений шага. Исследовать зависимость нормы погрешности от возмущения начальных условий при фиксированном шаге.
	
	\section{Постановка задачи}
	Дано ОДУ 2-ого порядка: $y'' + p(x)y' + q(x)y = f(x)$ \par
	$p(x) = - \tg{x}$\par
	$q(x) = 3$\par
	$f(x) = \sin{x}$\par
	Отрезок $[a;b]$: $a = 0, b = \frac{\pi}{2}$ \par
	Точное решение: $y=\sin{x}$\par
	n + 1 - число точек\par
	h - шаг, $h=\frac{b-a}{n}$\par
	Граничные условия задачи Коши: $y(x_0) = 0, y'(x_0) = 1$ \par
	Граничные условия краевой задачи: $y(x_0) = 0, y(x_n) = 1$ \par
	Задача: найти решения задачи Коши и краевой задачи для данного уравнения на данном отрезке.
	\begin{comment}
	Задача: найти $y_i^1 = y(x_i^1)$ - компоненты каркаса для численного решения задачи Коши и $y_i^2 = y(x_i^2)$ - компоненты каркаса для численного решения краевой задачи.
	\end{comment}   
	
	\section{Предварительный анализ задачи}
	Решением поставленных задач будем считать $y_i^1 = y(x_i^1)$ - компоненты каркаса для численного решения задачи Коши и $y_i^2 = y(x_i^2)$ - компоненты каркаса для численного решения краевой задачи. Непрерывную задачу сведем к дискретной. Разобьем отрезок $[a;b]$ n+1 точкой, в каждой из которых будем искать значение функции $y_i$. Для задачи Коши граничных условий достаточно, чтобы начать (используя разложение функции в ряд Тейлора до 4-й степени) последовательно находить значения функции в точках $x_i$, $i=0..n$. Для краевой задачи аппроксимируем $y''$ и $y'$, и подставляем полученные аппроксимации в исходное ОДУ, получая тем самым СЛАУ из n + 1 уравнения, где неизвестные - значения $y_i^2, i = 0..n$.
	
	\subsection{Алгоритмы}
	Для краевой задачи:
	
	а) Условия применимости: Потребуем для функций $p(x), q(x), f(x)$ непрерывности хотя бы из класса $C^{(0)}$. Гарантировать же применимость метода прогонки будет выполнение условия диагонального преобладания для матрицы коэффициентов: $|d_i| \geq |c_i| + |e_i|$, где $d_i, c_i, e_i$ - диагональные элементы, поддиагональные элементы и наддиагональные элементы соответственно. В частности, для метода конечных разностей и приведенного(!) ОДУ это условие сводится к следующему: 
	

	
	Первое условие для данной задачи приобретает вид: $h \leq 0.16$ (в дальнейшем это соответствует экспериментам)\par
	Второе условие выполняется для всех $"x"$ на отрезке.
	
	б) Этапы решения: Подставить аппроксимированные значения производных в исходное ОДУ, получив тем самым СЛАУ, первая и последняя строки которой определяются граничными условиями 1 типа. Решить систему методом прогонки. \par
	
	в) Алгоритм: Формулы аппроксимации:
	\begin{center}
		$y'(x_i) = \frac{y(x_{i+1}) - y(x_{i-1})}{2h} + O(h^2)$, $y''(x_i) = \frac{y(x_{i+1}) - 2y(x_i) + y(x_{i-1})}{h^2} + O(h^2)$ \\
	\end{center}
	\hspace{7mm} Полученное дискретное ОДУ:
	\begin{center}
		$\frac{y_{i+1} - 2y_i + y_{i-1}}{h^2} + \frac{y_{i+1} - y_{i-1}}{2h}p_i + q_iy_i = f_i$ \\
	\end{center}
	\hspace{7mm} Приведем подобные:
	\begin{center}
		$(1 - \frac{h}{2}p_i)y_{i-1} - (2 - h^2q_i)y_i + (1 + \frac{h}{2}p_i)y_{i+1} = h^2f_i$ \\
	\end{center}
	\hspace{7mm} Это уравнение заключает в себе строки СЛАУ для $i = 1..n-1$. Первая строка: $y_0 = A$, последняя строка: $y_n = B$. Решаем СЛАУ методом прогонки, т.е делаем прямой и обратный ход.\par
	\hspace{3mm} Прямой ход: в i-й строке выражаем $y_i$ и для $i = 1..n-1$ последовательно подставляем выраженные значения $y_{i-1}$.\par
	\hspace{3mm} Обратный ход: в i-ю строку подставляем значения $y_{i+1}$ для $i = n-1..1$, получая тем самым искомые значения $y_i, i = 0..n$.\par
	г) Теоретические расчёты: Метод конечных разностей изначально предполагается первого порядка. Однако порядок метода в данном случае определяется степенью аппроксимации производных при граничных условиях. Коль скоро эти условия 1-ого типа, говорить об аппроксимации не имеет смысла - они выполняются точно и порядок схемы полностью определяется порядком аппроксимации ДУ, и метод приобретает второй порядок, что в теории говорит о том, что при уменьшении величины шага на один порядок, точность должна увеличиваться на два.\par
	
	Для задачи Коши:
	
	а) Условия применимости: По теореме о существовании и единственности задачи Коши, если в некоторой окрестности точки $(x_0,y_0)$ $f(x)$ определена, непрерывна и имеет ограниченную по модулю производную, то существует окрестность $(x_0,y_0)$, в которой решение будет, и притом единственное. \par
	
	
	б) Этапы решения: идею методов Рунге-Кутты для ОДУ 1 порядка экстраполируем на ОДУ 2 порядка, произведя замену $z = y'$. Получим систему из двух уравнений. На каждом шаге будем вычислять 4 поправки, которые в своей линейной комбинации и будут составлять значение $y_i$ на каждом шаге.\par
	в) Алгоритм: Введем обозначения: $z = y'$, $z' = y'' = -p(x)y' - q(x)y + f(x) = F(x, y, z)$. Для каждого $x_i$ будем вычислять следующие выражения:\\
	$       \sigma_0 = F(x_i, y_i, z_i);\\
	\gamma_0 = z_i;\\ \\
	\sigma_1 = F(x_i + \frac{h}{2}, y_i + \gamma_0\frac{h}{2}, z_i + \sigma_0\frac{h}{2});\\
	\gamma_1 = z_0 + \sigma_0\frac{h}{2};\\ \\
	\sigma_2 = F(x_i + h/2, y_i + \gamma_1\frac{h}{2}, z_i + \sigma_1\frac{h}{2});\\
	\gamma_2 = z_i + \sigma_1\frac{h}{2};\\ \\
	\sigma_3 = F(x_i + h, y_i + h\gamma_2, z_i + h\sigma_2);\\
	\gamma_3 = z_i + h\sigma_2;$\par\par
	Причем $y_0$ и $z_0$ для $x_0$ задаются начальными условиями, а для последующих $x_i, i = 1..n$, $y_{i+1}$ и $z_{i+1}$ вычисляются по формулам со значениями $\sigma$ и $\gamma$ предыдущего шага: 

	
	
	г) Теоретические расчёты: Метод Рунге-Кутты локально имеет 5 порядок, но из-за накопления ошибки порядок снижается до 4-ого. Это означает, что в теории при уменьшении величины шага на один порядок должна прослеживаться тенденция увеличения точности на 4 порядка. При этом надо рассчитывать на то, что для точек, лежащих ближе к правому концу исходного отрезка, четвертый порядок наблюдать не придется.
	

	
	\section{Контрольные тесты}
	ОДУ 2-ого порядка: $y'' + p(x)y' + q(x)y = f(x), p(x) = - \tg{x}, q(x) = 3, f(x) = \sin{x}$\par
	а) Построим графики решений краевой задачи и задачи Коши для двух шагов: $h = 0.04$ и $h = 0.08$. Исходный отрезок: $[a;b], a = 0, b = \frac{\pi}{2}$. \par
	б) Измерим погрешность в зависимости от шага. Будем брать шаг $h = \frac{b-a}{2^{i+1}}, i = 0..13$ (14 значений) и получать бесконечную норму погрешности на отрезке $[a;b],\\ a = 0, b = \frac{\pi}{2}$. Найдем также координату максимальной погрешности на отрезке.
	
	\section{Численный анализ}	
	
	\section{Выводы}
	\begin{comment}
	Задача Коши и краевая задача - два частых явления, возникаюзих как результат обработки некоторой физической или математической модели. В основе каждой из них лежит идея решения ОДУ. Погрешности же решений в случае вышеупомянутых методов различаются, что является поводом к тому, чтобы понять, к какой лучше задаче сводить модель. К примеру, если отрезок, на котором необходимо вычислить решение ОДУ, велик, и не требуется точность более чем $10^{-3}$, то ресурсов на решение краевой задачи потребуется меньше, чем на решение задачи Коши при прочих равных. Тем не менее, если же требуется более высокая точность, то в рекомендацию всупает уже метод Рунге-Кутты 4 порядка и соответствующая ему задача Коши.
	\end{comment}
	
	
	Проводя сравнение погрешностей и координат максимальных ошибок для метода Рунге-Кутты и метода конечных разностей, можно говорить о следующем: для больших шагов (более чем $10^{-3}$) метод конечных разностей дает более точный результат. Тем не менее, для шагов, меньших чем $10^{-3}$, метод Рунге-Кутты явно лучше. Полезным может оказаться тот факт, что координаты максимальной ошибки у методов различны, что дает некоторую свободу при выборе метода.
\end{document}