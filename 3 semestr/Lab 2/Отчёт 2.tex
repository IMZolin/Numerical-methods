\documentclass{article}
\usepackage[T2A]{fontenc}
\usepackage[utf8]{inputenc}
\usepackage[russian]{babel}
\usepackage{alltt}
\usepackage{amsmath}
\usepackage{amsfonts}
\usepackage{indentfirst}

\title{non-linear equations}
\author{Иван Золин}
\date{November 2021}
\thispagestyle{empty}
\begin{document}
	
	\large
	\begin{center}
		
		Санкт-Петербургский политехнический университет\\
		Высшая школа прикладной математики и вычислительной физики, ФизМех
		
		~\\
		~\\
		~\\
		~\\
		Направление подготовки\\
		«01.03.02 Прикладная математика и информатика»
		
		~\\
		~\\
		~\\
		~\\
		Отчет по лабораторной работе № 2\\
		\textbf{тема "Решение систем линейных алгебраических уравнений.Прямые методы"}	
		~\\	Дисциплина "Численные методы"
	\end{center}
	
	~\\
	~\\
	~\\
	~\\
	~\\
	~\\
	~\\
	\begin{alltt}
		Выполнил студент гр. 5030102/00001			  		   		  Золин И.М.
		Преподаватель: 				              	        	Добрецова С.Б.
	\end{alltt}
	
	~\\
	~\\
	\begin{center}
		Санкт-Петербург
		
		~\\
		\textbf{2021}
	\end{center}{}
	
	\newpage
	
	\section{Формулировка задачи и её формализация}
		Дана система из n линейных уравнений с n неизвестными.
	$\sum\limits_{j=1}^{n}{a_{ij}x_j}=b_i, где i = 1,2,...,n$ , где $x_j$ неизвестные, $a_{ij}$ коэффициенты системы и $b_i$ компоненты вектора правой
	части.
	~\\
	В матричной форме $Ax = b$, где $A = (a_{ij}) \in \mathbb{R}^{n\times n} $ матрица коэффициентов, $b = (b_i) \in \mathbb{R}^n $вектор правой части и $x = (x_i) \in \mathbb{R}^n$  вектор неизвестных.
	Требуется найти x, прямой численный метод Вращения решения СЛАУ.
	
	Задача:
	
	1. Решить СЛАУ, используя метод вращения
	
	2. Построить график зависимости погрешности решения от числа обусловленности матрицы.
	\section{Алгоритм метода и условия его применимости}   
	\subsection{Алгоритм}
	Алгоритм делится на две подзадачи: прямой ход Методом вращений, с помощью которого приведем матрицу к верхнетреугольному виду и обратный ход Методом Гаусса, с помощью которого последовательно выразим переменные.
		\begin{enumerate}
			\item Прямой ход методом вращений:
			\begin{enumerate}
				
				\item Заменим i-ое уравнение линейной комбинацией	i-го и j-го уравнений, умноженных на $c_{ij}$ и $s_{ij}$ соответственно. Новое j-е уравнение - линейная комбинация i-го и j-го уравнений с коэффициентами $-s_{ij}$ и $c_{ij}$, , при $i = 1, 2,...,n$; $j = 2, 3,..,n; j > i$,, $k = 2, 3,..,n; k > i$ и $c^2 + s^2 = 1$
				
				$c = \frac{a_{ii}}{\sqrt{a_{ii}^2 + a_{ji}^2}}$
				
				$s = \frac{a_{ji}}{\sqrt{a_{ii}^2 + a_{ji}^2}}$
				
				$a_{ii} = c\cdot a_{ii} + s \cdot a_{ji} $
				
				$b_{i} = c\cdot b_{i} + s \cdot b_{j} $
				
				$b_{j} = -s\cdot b_{i} + c \cdot b_{j} $
				
				\item  Проводим действия для i-ых и j-ых столбцов.
				
				$a_{ik} = c\cdot a_{ik} + s \cdot a_{jk} $
				
				$a_{jk} = -s\cdot a_{ik} + c \cdot a_{jk} $
				
			В общем случае преобразования можно описать так в матричном виде:
				$A^{(1)}X = B^{(1)}$
				
				$A^{(1)}X = T_{1n}\cdot\cdot\cdot T_{13}T_{12}B^{(1)}$
				
				...
				
				$A^{(n-1)}X = B^{(т-1)} $
				
			где	$A^{(n-1)} = T_{n-1,n}A^{(n-2)}$, $b^{(n-1)} = T_{n-1,n}b^{(n-2)}$
				
				где \begin{equation*}
					T_{ij}(\varphi) = \left(
					\begin{array}{cccc}
						c_{ij} & s_{ij} & \ldots & 0\\
						-s_{ij} & c_{ij} & \ldots & 0\\
						\vdots & \vdots & \ddots & \vdots\\
						0 & 0 & \ldots & 0
					\end{array}
					\right)
				\end{equation*}
				
			\end{enumerate}	
		матрица поворота, задаваемая номерами исключаемого переменного i из уравнения j и углом
		поворота $\varphi$.
		
		Стоит отметить, что норма любого вектор-столбца расширенной матрицы системы остается
		такой же, как у соответствующего столбца исходной системы.
			\item Обратный ход метода Гаусса:
			
			$x_k = \frac{1}{a^{(n-1)}_{ii}}(b^{(n-1)}_i\sum\limits_{j=i+1}^{n}{a^{(n-1)}_{ij}x_j})$, где $x_i$ - соответственно i-я компонента искомого вектора X.
		\end{enumerate}
		
	\subsection{Условия применимости метода}
	\begin{enumerate}
		\item Требование: матрица коэффициентов СЛАУ не является вырожденной, т.е. $det(A) \neq 0$
		Это следует из условий построения матрицы $(A =Q \cdot D \cdot Q')$
		Q - ортогональная, а значит невырожденная матрица.
		D - диагональная, с ненулевыми элементами на диагонали - тоже невырожденная.
		Значит, их произведение - невырожденная матрица A.
		Тогда система имеет решение, причём единственное.
	\end{enumerate}
	\section{Предварительный анализ задачи}
	Для анализа зависимости погрешности решения от числа обусловленности нужно создать симметричные, положительно определённые матрицы 10x10. Для этого:
	\begin{enumerate}
		\item Заполняем диагональные элементы матрицы D от 1 до за-
		данного числа обусловленности.
		\item Находим ортогональную матрицу $Q = E - 2\frac{WW^T}{W^TW}$, где E - единичная матрица, $W \in M_{10 \times 1}$
		\item Находим матрицу $A = Q \cdot D \cdot Q'$
	\end{enumerate}
	В качестве решения берём столбец $x = (1,2,...,10)$, получаем столбец $b = Ax$, для которого мы знаем точное решение.
	
	\section{Тестовый пример с детальными расчётами для задачи малой размерности}
	\begin{equation*}
		\begin{cases}
			x + y + 2z = 5, 
			\\
			-x -y +2z = -1,
			\\
			x + z = 2.
		\end{cases}
	\end{equation*}
	Запишем в матричном виде, проведем прямой ход:
	\begin{enumerate}
\item	\[							
	\left(
	\begin{array}{ccc|c}
		1 & 1 & 2  & 5 \\
		-1 & -1 & 2  & -1 \\
		1 & 0 & 1  & 2 \\
	\end{array}
	\right)
	\] 
	
\item	\[							
	\left(
	\begin{array}{ccc|c}
		\sqrt{2} & \sqrt{2} & 0  & 3\sqrt{2} \\
		0 & 0 & 2\sqrt{2}  & 2\sqrt{2} \\
		1 & 0 & 1  & 2 \\
	\end{array}
	\right)
	\] 
	
\item	\[							
	\left(
	\begin{array}{ccc|c}
		\sqrt{3} & \frac{2}{\sqrt{3}} & \frac{1}{\sqrt{3}}  & \frac{8}{\sqrt{3}} \\
		0 & -\sqrt{\frac{2}{3}} & \sqrt{\frac{2}{3}}  & -\sqrt{\frac{2}{3}} \\
		0 & 0 & -2\sqrt{2}  & -2\sqrt{2} \\
	\end{array}
	\right)
	\] 
\end{enumerate}	
	
Теперь проведем обратный ход методом Гаусса:	
	\begin{enumerate}
		\item z = 1
		\item y = 2
		\item x+y+2z=5 
	\end{enumerate}
	Получили столбец-решение \[
	X = 
	\begin{pmatrix}
		1\\
		2\\
		1
	\end{pmatrix}
	\]
	\section{Подготовка контрольных тестов}
	 В качестве проверки полученных результатов строим график зависимости относительной погрешности $\frac{\|x -x^*\|}{\|x^*\|}$  вычислений от выбранного числа обусловленности cond, где $x^*$ - точное решение, сгенерированное в MATLAB, а x - решение,
	полученное реализованным методом на Си.
	
	\section{Модульная структура программы}
	GetMatrix(вх: conditionNumber, N; вых: matrix) Создаёт сим-
	метричную положительно определённую матрицу NxN с задан-
	ным числом обусловленности
	~\\
	~\\
	RotationMethod(вх: A, b; вых: x) Находит решение x системы
	Ax = b с помощью метода Вращения.
	
	\section{Анализ результатов}
	График зависимости нормы фактической ошибки решения СЛАУ
	и нормы невязки от числа обусловленности матрицы показыва-
	ет, что:
	
	1. С увеличением числа обусловленности норма фактической
	ошибки и норма невязки увеличивается
	
	2. При этом чаще норма фактической точности меньше нормы
	невязки, но в некоторых случаях норма невязки равна 0.
	
	График зависимости погрешности от ошибки во входных данных
	
	1.График показывает, что с увеличением ошибки во входных данных линейно возрастает относительная погрешность. Это ожидаемый результат, поскольку метод обладает хорошей обусловленностью.
	\section{Выводы}
	На основе полученных результатов можно сделать выводы,что с увеличением числа обусловленности входной матрицы растет и погрешность в вычислениях, также из положительных сторон метода вращения следует отметить, что при умножении матрицы поворота на вектор-столбец СЛАУ норма вектор-столбца остается равна исходной. Метод обладает хорошей обусловленностью. Из отрицательных сторон стоит отметить то, что время работы довольно долгое (даже дольше, чем метод Гаусса).
	
\end{document}