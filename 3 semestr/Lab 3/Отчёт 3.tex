\documentclass{article}
\usepackage[T2A]{fontenc}
\usepackage[utf8]{inputenc}
\usepackage[russian]{babel}
\usepackage{alltt}
\usepackage{amsmath}
\usepackage{amsfonts}
\usepackage{indentfirst}
\usepackage[dvips]{graphicx}
\graphicspath{{noiseimages/}}

\title{non-linear equations}
\author{Иван Золин}
\date{November 2021}
\thispagestyle{empty}
\begin{document}
	
	\large
	\begin{center}
		
		Санкт-Петербургский политехнический университет\\
		Высшая школа прикладной математики и вычислительной физики, ФизМех
		
		~\\
		~\\
		~\\
		~\\
		Направление подготовки\\
		«01.03.02 Прикладная математика и информатика»
		
		~\\
		~\\
		~\\
		~\\
		Отчет по лабораторной работе № 3\\
		\textbf{тема "Численные методы решение СЛАУ.Итерационные методы"}	
		~\\	Дисциплина "Численные методы"
	\end{center}
	
	~\\
	~\\
	~\\
	~\\
	~\\
	~\\
	~\\
	\begin{alltt}
		Выполнил студент гр. 5030102/00001			  		   		  Золин И.М.
		Преподаватель: 				              	        	Добрецова С.Б.
	\end{alltt}
	
	~\\
	~\\
	\begin{center}
		Санкт-Петербург
		
		~\\
		\textbf{2021}
	\end{center}{}
	
	\newpage
	
		\section{Формулировка задачи и её формализация}
		\subsection{Формулировка задания}
	Решить систему линейных уравнений методом Зейделя и исследовать зависимость нормы фактической ошибки, нормы невязки и числа итераций от заданной точности.
	~\\
	\subsection{Постановка задания}
	В матричной форме $Ax = b$, где $A = (a_{ij}) \in \mathbb{R}^{n\times n} $ матрица коэффициентов, $b = (b_i) \in \mathbb{R}^n $вектор правой части и $x = (x_i) \in \mathbb{R}^n$  вектор неизвестных.
	Требуется найти x, используя итерационный численный метод решения СЛАУ.
	

	\section{Алгоритм метода и условия его применимости}   
	Канонический вид линейных одношаговых итерационных методов:
	$B_k\frac{x^{(k+1)}-x^{(k)}}{\alpha_k} + Ax^{(k)} = b$,
	
	при $\alpha = \frac{1}{\|A\|}$ и $B_k = E$ получаем:
	
	$x^{(k+1)} = x^{(k)} - \frac{1}{\|A\|}(Ax^{(k)} - b)$
	
	\subsection{Алгоритм}
	\begin{enumerate}
		\item Действия до цикла:
		
		$g = \alpha_k \cdot b$, при $\alpha_k = \frac{1}{\|A\|}$ 
		
		$C =  E - \alpha \cdot A$
		
		%$x^{(k+1)} = C \cdot x^{(k)} + g$
		\item Цикл:
		Повторяй:
		
	%	$x = x_k$
		
		$x^{(k)}_i = g^{(k)}_i$, при $i = 1, 2,...,n$; $j = 1, 2,..,n$
		
		$t = i$
		
		Если $t > 0$, то
		
			$x^{(k)}_i = x^{(k)}_i + C_{ij} \cdot x^{(k)}_j$
		
			$t = t-1$
		
		Иначе
			
			$x^{(k)}_i = x^{(k)}_i + C_{ij} \cdot x^{(k-1)}_j$
		
		Пока: $\|x^{(k)} - x^{(k-1)}\|_{\infty} \geq \epsilon \frac{1 - \|C\|_{\infty}}{\|C\|_{\infty}}$
		
	\end{enumerate}
	
	\subsection{Условия применимости метода}
	Чтобы метод сходился необходимо:
	
	\begin{enumerate}
		\item $\|C\|_{1,2,\infty} < 1$, при $\sum\limits_{j=1}^{n}{|c_{ij}|},\|x_j\| = \max_i{|x_i|}$
		
		\item Все с.ч. $|\lambda_B| < 1,$ где $ B = (E - \bar C )^{-1} \cdot \bar \bar C$
		
	\end{enumerate}
	\section{Предварительный анализ задачи}
	Для анализа зависимости погрешности решения от числа обусловленности нужно создать симметричные, положительно определённые матрицы 10x10. Для этого:
\begin{enumerate}
	\item Находим ортогональную матрицу $Q = E - 2\frac{WW^T}{W^TW}$, где E - единичная матрица, $W \in M_{10 \times 1}$
	\item Находим матрицу $A = Q \cdot D \cdot Q'$
\end{enumerate}
В качестве решения берём столбец $x = (1,2,...,10)$, получаем столбец $b = Ax$, для которого мы знаем точное решение.

	\section{Тестовый пример с детальными расчётами для задачи малой размерности}
		\begin{equation*}
		\begin{cases}
			4x_1 - x_2 + x_3 = 4, 
			\\
			-x_1 +3x_2 -x_3 = 1,
			\\
			x_1 - x_2 +4x_3 = 4.
		\end{cases}
	\end{equation*}
	$
	a_{11} = 4> 1+1
	~\\
	a_{22} = 3> 1+1
	~\\
	a_{33} = 4> 1+1
	$
	$C = E- \alpha A, \alpha = \frac{1}{||A||}$
	
	$||C|| = \frac{37}{39} <1$
	Диагональное преобладание выполнено, значит можно использовать метод Зейделя.
	Начальное приближение: $x^{(0)} = (1; \frac{1}{3}; 1)$
	$x_1 = \frac{1}{4}(4+\frac{1}{3}-1) = 0.833
	~\\
	x_2 =  \frac{1}{3}(1 + 0.833 + 1) = 0.944
	~\\
	x_3 = \frac{1}{4}(4-0.833+0.944) = 1.028
	~\\
	x^{(1)} = (0.833 0.944 1.028)^T
	~\\
	x_1 =  \frac{1}{4}(4+0.944-1.028) = 0.979
	~\\
	x_2 =  \frac{1}{4}(1+0.979+1.028) = 1.002
	~\\
	x_3 =  \frac{1}{4}(4-0.979+1.002) = 1.0063
	~\\
	x^{(2)} = (0.979 1.002 1.006)^T
	~\\
	x_1 =  \frac{1}{4}(4+1.002-1.006) = 0.999
	~\\
	x_2 = \frac{1}{4}(4+0.999+1.006) = 1.002
	~\\
	x_3 = \frac{1}{4}(1-0.999+1.002) = 1.001
	~\\
	x^{(3)} = (0.999; 1.002; 1.001)^T
	~\\
	x_1 = \frac{1}{4}(4+1.002-1.001) = 1
	~\\
	x_2 = \frac{1}{3}(1 + 1 + 1.001) = 1
	~\\
	x_3 = \frac{1}{4}(4-1+1) = 1
	~\\
	x^{(4)} = (1;1;1)^T
	$
	\section{Подготовка контрольных тестов}
	\paragraph{}Создаётся положительно определенная матрица  $A_{10 \times 10}$  для нахождения решения СЛАУ методом Зейделя с заданной точностью  $\epsilon=10^{-i}$, где $i \in [3, 7], i \in \mathbb {N} $. 
	
	
	\section{Модульная структура программы}
	SeidelMethod(вх: $A,b,\epsilon$, вых: $x$,n) Находит вектор решений x СЛАУ методом Зейделя c точностью $\epsilon$ за n итераций. A - матрица коэф-тов, b - столбец свободных коэф-тов.
	
	InfNorm(вх: $matrix$, вых: $norm$) Находит бесконечную норму матрицы $matrix$
	
	Norm(вх:$matrix$, вых:$norm$) Находит вторую норму матрицы $matrix$
	
	GetFinalMatrix(вх:$size$, вых:$A$) Создаёт матрицу $A$

	\section{Анализ результатов}
		1. График зависимости нормы фактической точности и нормы невязки от заданной точности показывает, что с увеличением точности монотонно уменьшаются нормы.
		
		2. График зависимости числа итераций от заданной точности показывает, что с увеличением точности возрастает количество необходимых итераций для её достижения.


	\section{Выводы}
		Метод Зейделя решения СЛАУ довольно прост в реализации, гарантированно находит решение с заданной точностью и не требует больших вычислительных затрат. Однако он накладывает большие условия применимости на матрицу коэффициентов,которая должна быть симметричной и положительно определенной.
	
\end{document}