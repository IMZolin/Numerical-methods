\documentclass{article}
\usepackage[T2A]{fontenc}
\usepackage[utf8]{inputenc}
\usepackage[russian]{babel}
\usepackage{alltt}
\usepackage{amsmath}
\usepackage{amsfonts}
\usepackage{indentfirst}

\title{non-linear equations}
\author{Иван Золин}
\date{November 2021}
\thispagestyle{empty}
\begin{document}
	
	\large
	\begin{center}
		
		Санкт-Петербургский политехнический университет\\
		Высшая школа прикладной математики и вычислительной физики, ФизМех
		
		~\\
		~\\
		~\\
		~\\
		Направление подготовки\\
		«01.03.02 Прикладная математика и информатика»
		
		~\\
		~\\
		~\\
		~\\
		Отчет по лабораторной работе № 4\\
		\textbf{тема "Алгебраическая проблема собственных значений" }	
		~\\	Дисциплина "Численные методы"
	\end{center}
	
	~\\
	~\\
	~\\
	~\\
	~\\
	~\\
	~\\
	\begin{alltt}
		Выполнил студент гр. 5030102/00001			  		   		  Золин И.М.
		Преподаватель: 				              	        	Добрецова С.Б.
	\end{alltt}
	
	~\\
	~\\
	\begin{center}
		Санкт-Петербург
		
		~\\
		\textbf{2021}
	\end{center}{}
	
	\newpage
	
	\section{Формулировка задачи и её формализация}
	\subsection{Формулировка задания}
	\begin{enumerate}
		\item Найти два максимальных по модулю, но разных по знаку собственных числа, используя метод скалярных произведений.
		\item C помощью метода обратных итераций со сдвигом уточнить
		собственные числа и найти собственные векторы матрицы.
		\item Исследовать зависимости нормы фактической точности для
		собственных чисел и собственных векторов, нормы невязки $||Ax - \lambda x||$
		и числа итераций от заданной точности.
	\end{enumerate}
	\subsection{Постановка задания}
	Дана матрица $A \in \mathbb{R}^{n \times n}$. Если выполнено условие $A \cdot X = \lambda \cdot X$,где X - ненулевой вектор, $\lambda$ - это  число, то Х – это собственный вектор (СВ) матрицы А. $\lambda$ - это собственное число (СЧ) матрицы А, соответсвующее СВ X.
	
	$A \cdot X = \lambda \cdot X \Leftrightarrow \det(A-\lambda E) = 0 \Leftrightarrow p_0\lambda^n + p_1\lambda^{(n-1)}+...+p_n$
	
	Из основной теоремы алгебры следует, что матрица А имеет ровно n СЧ.
	
	Требуется решить частичную алгебраическую проблему собственных значений: найти два максимальных по модулю, но разных по знаку собственных числа, используя метод скалярных произведений.
\newpage
	
		\section{Алгоритм метода и условия его применимости}   
	
	\subsection{Метод скалярных произведений}
	\subsubsection{Алгоритм}
	1. Входные данные: матрица А, заданная точность $\epsilon$
	
	2. Цикл:
	
	k = 1
	
	Повторяй:
	$
	~\\
	y^{(k)} = Ax^{(k-1)}
	~\\
	z^{(k)} = A y^{(k)}
	~\\
	\lambda^{(k)} = \sqrt{\frac{(z^{(k)}, x^{(k-1)})}	{(x^{(k-1)}, x^{(k-1)})}}	
	~\\
	x^{(k)} = \frac{y^{(k)}}{||y^{(k)}||_2}
	~\\
	w^{(k)}_{1} = z^{(k)} + \lambda\cdot  y^{(k)}
	~\\
	w^{(k)}_{2} = z^{(k)} - \lambda\cdot y^{(k)}
	~\\
	k++
	~\\	
	$
	
	Пока:$\|\lambda^{(k)} - \lambda^{(k-1)}\|> \epsilon $
	
	Конец
	
	3. Результат: $\lambda = \lambda^{(k)}, w_1 = w^{(k)}_{1}, w_2 = w^{(k)}_{2} $
	~\\
	
	
	
	\subsubsection{Условия применимости метода}
	Пусть $A \in \mathbb{R}^{n \times n}$, а $\lambda_1, \lambda_2,...,\lambda_n$ - собственные числа, исследуемой матрицы A, причём $\lambda_1 = -\lambda_2$ и $|\lambda_1| = |\lambda_2| > |\lambda_3| \geq |\lambda_4| \geq ... \geq |\lambda_n|$, матрица А - матрица простой структуры и симметричная. Тогда метод скалярных произведений сходится.

	
	
	\section{Предварительный анализ задачи}
	Согласно теореме: если матрица вещественная и симметричная, то она является матрицей простой структуры. Будем задавать матрицы, удовлетворяющие данному критерию.
	
	Такие матрицы будем искать с помощью QR-разложения из пакета Matlab, где матрица Q – ортогональная, R - верхнетреугольная.(Генерируются матрица, полученную матрицу представляем в виде QR-разложения, т.е. умножения 2 матриц Q и R). С помощью QR-разложения находим матрицу Q.
	
	Задается диагональная матрица D, у нее на диагонали – собственные числа. 
	
	Нужная симметричная матрица получается следующим образом:
	$A = Q \cdot D \cdot Q^T$
	
	\section{Тестовый пример с детальными расчётами для задачи малой размерности}
	  
	  
	  
	\[A =							
	\left(
	\begin{array}{ccc}
		-0.5222 & 0.8527 & -0.0134\\
		0.8428 & 0.5135 & -0.1613 \\
		0.1306 & 0.0955 & 0.9868
	\end{array}
	\right)
	\] 

	Точные собственные числа:
	$\lambda_{1,2} = \pm 7 , \lambda_3 = 1$
	
	Точные собственные векторы:

	$X_1 = \begin{pmatrix}
		0.8527\\
		0.5135\\
		0.0955\\
	\end{pmatrix}$,
	$X_2 =\begin{pmatrix}
	-0.5222\\
	0.8428\\
	0.1306
	\end{pmatrix}$,
	
	$X_3 =\begin{pmatrix}
		-0.0134\\
		-0.1613\\
		0.9868
	\end{pmatrix}$,
	
%	Метод скалярных произведений:
	Начальное приближение:
	$x^{(0)}=\begin{pmatrix}
		\frac{1}{\sqrt{3}}\\
		\frac{1}{\sqrt{3}}\\
		\frac{1}{\sqrt{3}}
	\end{pmatrix}$,
		~\\
		
	1-ая итерация:
	
	\[y^{(1)} = A \cdot x^{(0)}							
	\left(
	\begin{array}{ccc}
		5.9842 \\
		1.4214 \\
		0.7890
	\end{array}
	\right) \] 
	
	\[z^{(1)} = A \cdot y^{(0)}							
	\left(
	\begin{array}{ccc}
		28.5948 \\
		31.9222 \\
		6.0812
	\end{array}
	\right) \] 
	
	$\|y^{(1)}\| = 6.2011$
	
	%$({y^{(1)}},{x^{(-1)}}) = 6.8493$
	
	$\lambda^{(1)} = \sqrt{\frac{(z^{(1)}, x^{(0)})}	{(x^{(0)}, x^{(0)})}} =  6.2006$
	
	
	\[x^{(1)} = \frac{y^{(1)}}{||y^{(1)}||}= 							
	\left(
	\begin{array}{ccc}
		0.9650  \\
		0.2292  \\
		0.1272
	\end{array}
	\right) \] 
	
	\[w^{(1)}_1 =z^{(1)} +  \lambda y^{(1)}=							
	\left(
	\begin{array}{ccc}
		0.8415 \\
		0.5217 \\
		0.1405
	\end{array}
	\right) \]
	
	\[w^{(1)}_2 =z^{(1)} - \lambda y^{(1)}=							
	\left(
	\begin{array}{ccc}
		-0.3452 \\
		0.9373 \\
		0.0482 
	\end{array}
	\right) \]
	
	
	~\\
	2-ая итерация:
	\[y^{(2)} = A \cdot x^{(1)}							
\left(
\begin{array}{ccc}
	4.6112 \\
	5.1478 \\
	0.9807
\end{array}
\right) \] 

\[z^{(2)} = A \cdot y^{(1)}							
\left(
\begin{array}{ccc}
	47.3347 \\
	11.8167 \\
	2.6522
\end{array}
\right) \] 

$\|y^{(2)}\| = 6.9803$

%$({y^{(1)}},{x^{(-1)}}) = 6.8493$

$\lambda^{(1)} = \sqrt{\frac{(z^{(2)}, x^{(1)})}	{(x^{(1)}, x^{(1)})}} =  6.9803$


\[x^{(2)} = \frac{y^{(2)}}{||y^{(2)}||}= 							
\left(
\begin{array}{ccc}
	0.6606  \\
	0.7375  \\
	0.1405
\end{array}
\right) \] 

\[w^{(2)}_1 =z^{(2)} +  \lambda y^{(2)}=							
\left(
\begin{array}{ccc}
	0.8529 \\
	0.5121 \\
	0.1019
\end{array}
\right) \]

\[w^{(2)}_2 =z^{(2)} - \lambda y^{(2)}=							
\left(
\begin{array}{ccc}
	0.5262 \\
	-0.8378 \\
	-0.1457 
\end{array}
\right) \]
	~\\
	
	3-ья итерация:
	\[y^{(3)} = A \cdot x^{(2)}							
\left(
\begin{array}{ccc}
	6.7811 \\
	1.6929 \\
	0.3800
\end{array}
\right) \] 

\[z^{(3)} = A \cdot y^{(2)}							
\left(
\begin{array}{ccc}
	32.3763 \\
	36.2199 \\
	6.3708
\end{array}
\right) \] 

$\|y^{(3)}\| = 6.9996$

%$({y^{(1)}},{x^{(-1)}}) = 6.8493$

$\lambda^{(3)} = \sqrt{\frac{(z^{(3)}, x^{(2)})}	{(x^{(2)}, x^{(2)})}} =  6.9996$


\[x^{(3)} = \frac{y^{(3)}}{||y^{(3)}||}= 							
\left(
\begin{array}{ccc}
	0.9688  \\
	0.2419  \\
	0.0543
\end{array}
\right) \] 

\[w^{(3)}_1 =z^{(3)} +  \lambda y^{(3)}=							
\left(
\begin{array}{ccc}
	0.8527 \\
	0.5134 \\
	0.0964
\end{array}
\right) \]

\[w^{(3)}_2 =z^{(3)} - \lambda y^{(3)}=							
\left(
\begin{array}{ccc}
	-0.5221 \\
	0.8432 \\
	0.1284 
\end{array}
\right) \]

	
	\section{Подготовка контрольных тестов}
	\paragraph{}Создаётся симметричная матрица  $A_{10 \times 10}$  для нахождения собственных чисел с помощью метода скалярных произведений с точностью  $\epsilon=10^{-i}$, где $i \in [0, 15], i \in \mathbb {N} $. 
	Собственные числа:[-50, 50, 1, 8, 2, 7, 6, 5, 3, 4]
	
	\section{Модульная структура программы}
	ScalarProductMethod(вх: $A, \epsilon$, вых: $\lambda_{1,2}, w_{1,2}, n$) Находит  максимальные собственные числа разные по знаку матрицы А  $\lambda_{1,2}$ и соответствующие им собственные вектора $w_{1,2}$ с помощью метода скалярных произведений за n итераций с заданной точностью $\epsilon$.

	
	\section{Анализ результатов}
	1.По графику зависимости количества итераций метода скалярных произведений от заданной точности видно, что количество итераций увеличивается линейно [3 4 4 5 5 6 6 7 8 8 9 9 10 11 13 ].
	При изменении $\epsilon$ на порядок количество итераций изменяется примерно на 1. 
	
	2. Графики заивисимости с.в. и невязок от заданной точности параллельны. Также все графики достигают минимум точности $\epsilon = 10^{-11}$.
	
	
	\section{Выводы}
	С помошью метода скалярных произведений достигается заданная точность нахождения собственных чисел, при хорошей отделимости.
	Также с хорошей отделимостью требуется меньшее количество итераций для нахождения собственных чисел и собственных векторов.
	\end{document}