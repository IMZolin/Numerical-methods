\documentclass{article}
\usepackage[T2A]{fontenc}
\usepackage[utf8]{inputenc}
\usepackage[russian]{babel}
\usepackage{alltt}
\usepackage{amsmath}
\usepackage{amsfonts}
\usepackage{indentfirst}

\title{non-linear equations}
\author{Иван Золин}
\date{October 2021}
\thispagestyle{empty}
\begin{document}
	
	\large
	\begin{center}
		
		Санкт-Петербургский политехнический университет\\
		Высшая школа прикладной математики и вычислительной физики, ФизМех
		
		~\\
		~\\
		~\\
		~\\
		Направление подготовки\\
		«01.03.02 Прикладная математика и информатика»
		
		~\\
		~\\
		~\\
		~\\
		Отчет по лабораторной работе № 1\\
		\textbf{тема "Решение алгебраических и трансцендентных уравнений"
			дисциплина "Численные методы"}
	\end{center}
	
	~\\
	~\\
	~\\
	~\\
	~\\
	~\\
	~\\
	\begin{alltt}
		Выполнил студент гр. 5030102/00001			  		   		  Золин И.М.
		Преподаватель: 				              	        	Добрецова С.Б.
	\end{alltt}
	
	~\\
	~\\
	\begin{center}
		Санкт-Петербург
		
		~\\
		\textbf{2021}
	\end{center}{}
	
	\newpage
	
	\section{Формулировка задачи и её формализация}
	\subsection{Формулировка задания}
	Решить алгебраическое $2x^4-x^2-10=0$ и трансцедентное $x + \lg{(1+x)}-1.5=0$ уравнения методом секущих и методом половинного деления и исследовать зависимость количества итераций от заданной точности.
	
	\subsection{Постановка задания}
	Дано $f(x)=0$, $x\in[a, b]$, на котором $\exists!x*:f(x*)\equiv0$. Найти $x\in[a, b]:|x-x*|<\epsilon$, где $\epsilon$ - заданная точность.
	
	\section{Алгоритм методов и условия их применимости}
	\subsection{Метод половинного деления}
	\subsubsection{Алгоритм}
	\begin{enumerate}
		\item Входные данные: функция $f(x)$ на $[a, b]$, абсолютная погрешность $\epsilon$.
		\item Цикл: 
		\begin{enumerate}
			\item Вычислить $x=\frac{a+b}{2}$.
			\item Если $f(x)f(a)<0$, то $b=x$, иначе $a=x$.
			\item Проверка на точность: если $|b-a|>2\epsilon$, перейти к итерации (а) цикла 
		\end{enumerate}
		\item Результат: Положить $x^*=\frac{a+b}{2}$
	\end{enumerate}
	\subsubsection{Условия применимости метода:}
	\begin{enumerate}
		\item $f \in C([a, b])$
		\item $f(a)f(b)<0$
	\end{enumerate}
	~\\    
	\subsection{Метод простых итераций}
	\subsubsection{Алгоритм}
	\begin{enumerate}
		\item Входные данные: функция $\varphi(x)$ на $[a, b]$, параметр $q$, зависящий от $\varphi(x)$, абсолютная погрешность $\epsilon$.
		\item Действия до цикла:
		Построить последовательность $\{x_k\}$ по формуле  $x_k = \varphi(x_{k-1})$, где $\varphi(x)$ такая, что $f(x) = 0 \iff x = \varphi(x)$
		\item Цикл: 
		\begin{enumerate}
			\item Положить $x_k = \varphi(x_{k-1})$.  
			\item Проверка на точность: если $|x_k-x_{k-1}| < \frac{1-q}{q}\epsilon$, перейти к итерации (а) цикла .
		\end{enumerate}
		\item Результат: Положить $x^*=x_k$, при $|x_{k+1}-x_k| \geq \frac{1-q}{q}\epsilon$
	\end{enumerate}
	\subsubsection{Условия применимости метода}
	\begin{enumerate}
		\item $\varphi \in C^1([a, b])$
		\item $\varphi(x)\in [a, b]$ для $\forall x \in [a, b]$
		\item $\exists q: |\varphi'(x)|\leq q<1$ для $\forall x \in [a, b]$
	\end{enumerate}
	\section{Предварительный анализ задачи}
	\subsection{Теорема о верхней границе}
	Для уравнения $2x^4-x^2-10=0$ найдём отрезки, содержащие все корни, применив \emph{Теорему о верхней границе положительных корней} 4 раза: $x^*<\sqrt[m](\frac{|a'|}{a_0})$, где m - номер первого отрицательного коэффициента, $a_0$ - первый коэффициент, а a' - наибольший по модулю отрицательный коэффициент.
	\begin{enumerate}
		\item $2x^4-x^2-10=0 \Rightarrow m=2, a'=-10, a_0=2 \Rightarrow
		\\ \Rightarrow x^* \leq 1 + \sqrt{5}\approx 3.236$
		\item $x=(\frac{1}{y}) \Rightarrow 10y^4+y^2-2=0 \Rightarrow m=2, a'=2, a_0=10 \Rightarrow y \leq 1 + \frac{1}{\sqrt5} \Rightarrow x^* \leq \frac{5-\sqrt{5}}{4}\approx 0.69 $
		\item $x=-y \Rightarrow 2y^4-y^2-10=0 \Rightarrow m=2, a'=-10, a_0=2 \Rightarrow
		\\ \Rightarrow y \leq 1 + \sqrt{5} \Rightarrow x^* \geq -1 - \sqrt{5} \approx -3.236$
		\item $x=-1/y \Rightarrow 10y^4+y^2-2=0 \Rightarrow m=2, a'=2, a_0=10 \Rightarrow y \leq 1 + \frac{1}{\sqrt5} \Rightarrow x^* \geq \frac{\sqrt{5}-5}{4} \approx -0.69 $
	\end{enumerate}
	\subsection{Нахождение отрезка с одним корнем и проверка условий применимости методов}
	\subsubsection{Алгебраическое уравнение}
	С помощью MATLAB устанавливаем, что на отрезке $[1, 2]$ уравнение $2x^4-x^2-10=0$ имеет единственный корень и прояверяем \emph{условия применимости}.
	~\\
	~\\
	Метод половинного деления:
	\begin{enumerate}
		\item Функция непрерывна на всей числовой оси(а значит и на отрезке)
		\item $f(1)f(2)=-9*18 = -162 < 0$
	\end{enumerate}
	Метод простых итераций:
	По \emph{ Теореме 2 (о сходимости метода простых итераций)}, проверим  условие: 
	$\exists q: |\varphi'(x)|\leq q<1$ для $\forall x \in [a, b]$
	~\\
	~\\
	$x = \varphi(x)$ , $x = x - \alpha*f(x) \Rightarrow \varphi(x) = x - \alpha * f(x)$
	~\\
	~\\
	
	Проверка  условия:  
	\begin{enumerate}
	\item $f(x) = 2x^4 -x^2 -10\Rightarrow f'(x) = 8x^3 -2x \Rightarrow f''(x) = 24x^2 -2$.
	\item Корни второй производной $f''(x)$, $ x = \pm \frac{\sqrt{3}} {6}$ не пренадлежат отрезку$[1, 2] \Rightarrow f'(x)$ монотонна на отрезке $[1, 2]$.
	\item $f'(1) = 6; f'(2) = 60 \Rightarrow \max_{x \in{[1, 2]}} |f'(x)| = f'(2) = 60 \Rightarrow M_1 = 60 $.
	\item $|\alpha| < \frac{2} {M_1} = \frac{1}{30}$
	\item $\varphi(x) = x - \alpha*f(x) = x -  \frac{1}{30} * f(x) = x - \frac{1}{30} * (2x^4 - x^2 -10) $
	\item $|\varphi'(x)|\leq q<1 \Rightarrow \varphi'(x) = \frac{-4}{15}*x^3 + \frac{1}{15}*x +1 $
	~\\
	$\varphi'(1) = \frac{12}{15} = 0.8 \Rightarrow$ условие $|\varphi'(1)|\leq q<1$ выполняется, значит $\exists q: |\varphi'(x)|\leq q<1$ для $\forall x \in [a, b]$.
	\end{enumerate}


	\subsubsection{Трансцендентное уравнение}
	С помощью MATLAB устанавливаем, что на отрезке $[0.7, 1.7]$ уравнение $x + \lg{(1+x)}-1.5=0$ имеет единственный корень и проверяем \emph{условия применимости}.
	~\\
	~\\
	Метод половинного деления:
	\begin{enumerate}
		\item Функция непрерывна на всей числовой оси(а значит и на отрезке)
		\item $f(0.7)f(1.7)\approx -0.359594 < 0$
	\end{enumerate}
	Метод простых итераций:
	По \emph{ Теореме 2 (о сходимости метода простых итераций)}, проверим условие: 
		$\exists q: |\varphi'(x)|\leq q<1$ для $\forall x \in [a, b]$

	$x = \varphi(x)$ , $x = x - \alpha*f(x) \Rightarrow \varphi(x) = x - \alpha * f(x)$
	~\\
	$|\alpha| < \frac{2} {M_1}$, где $M_1 = \max_{x \in{[a, b]}} |f'(x)|$
	~\\
	~\\
	
	Проверка условия:  
	\begin{enumerate}
		\item $f(x) = x + \lg{(1+x)}-1.5\Rightarrow f'(x) = \frac{1}{\ln(10)(1+x)} +1 \Rightarrow f''(x) = \frac{-1}{\ln(10)(1+x)^2}$.
		\item Корней второй производной $f''(x)$ не существует $ \Rightarrow f'(x)$ монотонна на отрезке $[0.7, 1.7]$.
		\item $f'(0.7) = \frac{1}{\ln(10)(1.7)} \approx 1.255467; f'(1.7) = \frac{1}{\ln(10)(2.7)} \Rightarrow \max_{x \in{[0.7, 1.7]}} |f'(x)| = f'(0.7) \approx 1.255467 \Rightarrow M_1 = 1.255467 $.
		\item $|\alpha| < \frac{2} {M_1} = \frac{2}{1.255467} = 1.6$
		\item $\varphi(x) = x - \alpha*f(x) = x -  1.6 * f(x) = x - 1.6 * (x + \lg{(1+x)}-1.5) = - 0.6*x -1.6*\lg{(1+x)} + 2.4$
		\item $|\varphi'(x)|\leq q<1 \Rightarrow \varphi'(x) = - 0.6 +\frac{-1.6}{\ln(10)(x+1)} $
		~\\
		$\varphi'(1.7) = - 0.6 +\frac{-1.6}{\ln(10)(2.7)} = -0.85 \Rightarrow$ условие $|\varphi'(1)|\leq q<1$ выполняется, значит $\exists q: |\varphi'(x)|\leq q<1$ для $\forall x \in [a, b]$.
	\end{enumerate}
	
	\section{Тестовый пример с детальными расчётами для задачи малой размерности}
	$$f(x)=x^2 + 10x$$
	Очеивдно, что существует два корня $x = 0$ и $x = -10$. Найдем на отрезке $[-2,1]$ корень этого уравнения, используя численный метод половинного деления. 
	~\\
	На этом отрезке функция непрерывна, а также $f(-1)*f(2) = -176 < 0$.
	~\\
	\begin{enumerate}
		\item $\frac{b-a}{2} = 1.5$
		~\\
		$c=\frac{a + b}{2}=\frac{-2 + 1}{2} = -0.5$
		~\\
		$f(a)*f(c) = 76 > 0 \Rightarrow a = c$
		\item $\frac{b-a}{2} = 0.75$
		~\\
		$c=\frac{a + b}{2}=\frac{-0.5 + 1}{2} = 0.25$
		~\\
		$f(a)*f(c) = -12 < 0 \Rightarrow b = c$
		\item $\frac{b - a}{2} = 0.375$
		~\\
		$c=\frac{a + b}{2}=\frac{-0.5 + 0.25}{2} = -0.125$
		~\\
		$f(a)*f(c) = 6 > 0 \Rightarrow a = c$
		\item $\frac{b - a}{2} = 0.1875$
		~\\
		$c=\frac{a + b}{2}=\frac{-0.125 + 0.25}{2} = 0.0625$
		~\\
		$f(a)*f(c) = -1 < 0 \Rightarrow b = c$
		\item $\frac{b - a}{2} = 0.09375$
		~\\
		$c=\frac{a + b}{2}=\frac{-0.125 + 0.0625}{2} = -0.03125$
		~\\
		$f(a)*f(c) = 0$
		~\\
		$x \approx -0.03125$		
	\end{enumerate}
	
	\paragraph{}Найдем на том же самом отрезке корень $x \in [-2, 1]$, но с использованием метода простых итераций.
	По \emph{ Теореме 2 (о сходимости метода простых итераций)}, проверим условие: 
	\begin{enumerate}
		\item $\exists q: |\varphi'(x)|\leq q<1$ для $\forall x \in [a, b]$
	\end{enumerate}
	~\\	
	$x = \varphi(x)$ , $x = x - \alpha*f(x) \Rightarrow \varphi(x) = x - \alpha * f(x)$
	~\\
	$|\alpha| < \frac{2} {M_1}$, где $M_1 = \max_{x \in{[a, b]}} |f'(x)|$
	~\\
	~\\
	Проверка условия $\exists q: |\varphi'(x)|\leq q<1$ для $\forall x \in [a, b]$: найдём $\varphi(x)$ и $ q$.  
	\begin{enumerate}
		\item $f(x) = x^2 + 10x \Rightarrow f'(x) = 2x + 10 $ - линейная монотонно возрастающая функция.
		\item $f'(1) = 12 \Rightarrow \max_{x \in{[-2, 1]}} |f'(x)| = f'(1) = 12 \Rightarrow M_1 = 12$.
		\item $|\alpha| < \frac{2} {M_1} = \frac{1}{6} \approx 0.167$
		\item $\varphi(x) = x - \alpha*f(x) = x -  \frac{1}{6} * f(x) = x - \frac{1}{6} * (x^2 + 10x) $
		\item $|\varphi'(x)|\leq q<1 \Rightarrow \varphi'(x) = -  \frac{1}{3}*x - \frac{1}{3} $
		~\\
		$\varphi'(1) = \frac{-4}{3} \approx -1.33;$ $ \varphi'(-2) = \frac{1}{3} \approx 0.33 \Rightarrow$ условие $|\varphi'(-2)|\leq q<1$ выполняется, значит $\exists q: |\varphi'(x)|\leq q<1$ для $\forall x \in [a, b]$.
	\end{enumerate}
	Алогритм для $\epsilon = 0.1$,получается $\frac{1-q}{q}\epsilon = 0.2$:
	\begin{enumerate}
	\item $x_k = a = -2$	
	\item $x_k = \varphi(x_{k-1}) = \frac{1}{3}$
	~\\ $|x_k - x_{k-1}| = \frac{7}{3}$
	\item $x_k = \varphi(x_{k-1}) = \frac{-4}{9}$ 
	~\\	$|x_k - x_{k-1}| = \frac{-7}{9}$
	\item $x_k = \varphi(x_{k-1}) = \frac{-5}{27}$ 
	~\\	$|x_k - x_{k-1}| = \frac{7}{27}$
	\item $x_k = \varphi(x_{k-1}) = \frac{-22}{81}$ 
	~\\	$|x_k - x_{k-1}| = \frac{-7}{81} \approx 0.086$
	\end{enumerate}	
	\section{Подготовка контрольных тестов}
	\paragraph{}Методы половинного деления и секущих использовались для нахождения корней алгебраического $2x^4-2x^2-10=0$ уравнения на отрезке $[1,2]$ с найденной функцией$\varphi(x) = \frac{-1}{15}*x^4 + \frac{1}{30}*x^2 + x + \frac{1}{3}$ для метода простых итераций и трансцедентного $x + \lg{(1+x)}-1.5=0$ уравнения на отрезке $[0.7,1.7]$ с найденной функцией$\varphi(x) = - 0.59303*x -1.59303*\lg{(1+x)} + 2.3895$ для метода простых итераций с точностью $\epsilon=10^{-i}$, где $i \in [1, 6], i \in \mathbb {N} $. Также использовалась
	функция MATLAB fzero с такой же точностью.

	\section{Модульная структура программы}
	BisectionMethod(вх: $f,a,b,\epsilon$, вых: $x^*$,n) Находит корень $x^*$ уравнения $f(x)=0$ на отрезке $[a, b]$ с помощью метода половинного деления точностью $\epsilon$ за n итераций.
	~\\
	~\\
	FixedPointIterations(вх: $\varphi$,$x_0,x_1,q,\epsilon$, вых: $x^*$,n) Находит корень $x^*$ уравнения $f(x)=0$ на отрезке $[a, b]$ с помощью функции $\varphi(x)$, указатель на которую подаётся в функцию, и параметр q, заранее вычисленные,  с точностью $\epsilon$ за n итераций.
	\section{Анализ результатов}
	\paragraph{}Графики зависимости количества итераций методов от точности для алгебраической и трансцедентной функции показывают, что: 
	\begin{enumerate}
	\item Метод fzero работает лучше всего для обоих функций. 
	\item Также зависимости в обоих случиях линейные, что для алгеброической(для fzero: 2, 5, 5, 6, 6, 7; для метода половинного деления: 3, 6, 9, 13, 16, 19; для метода простых итераций: 4, 5, 6, 7, 7, 8), то и для трансцедентной(для fzero: 2, 5, 5, 6, 6, 7; для метода половинного деления: 3, 6, 9, 13, 16, 19; для метода простых итераций: 45, 70, 95, 120, 146, 171), поэтому с увеличением точности $\epsilon$ увеличивается количество итераций, потому что алгоритмы этих методов подразумевают, что с увеличением $\epsilon$ будет дольше в цикле соблюдаться условие выхода из него.
	\end{enumerate}
	\paragraph{}Графики зависимости количества итераций методов от точности для алгебраической функции показывают, что:
	\begin{enumerate}
		\item После метода fzero по быстроте нахождения корня с заданной точностью идёт метод простых итераций , а затем метод половинного деления, т.е. требует заметно большего количества итераций. До значения $\epsilon=10^{-2}$ быстрее метода простых итераций оказывается метод половинного деления.  
		
		\item В точке $(\epsilon=10^{-2},$ 5 итераций) графики методов fzero и метода простых итераций пересекаются. Несложно заметить, что для метода половинного деления с $\epsilon=10^{-3}$ наблюдается ухудшение работы , из-за увеличения количества итераций. А в $\epsilon=10^{-6}$ метод fzero имеет 7 итераций, метод простых итераций 9, метод половинного деления 19.
	\end{enumerate}
	\paragraph{}Графики зависимости количества итераций методов от точности для трансцедентнойй функции показывают, что:
 \subparagraph{}
 После метода fzero по быстроте нахождения корня с заданной точностью идёт метод половинного деления немного, затем метод простых итераций. Метод простых итераций для трансцедентной функции справляется хуже всего.


	\paragraph{}График зависимости модуля значений  алгебраической и трансцедентной функции от заданной точности показывает, что:
	\begin{enumerate}
		\item Значение корня, полученного с помощью метода fzero, с увеличением точности стремиться к нулю монотонно.  
		\item Значения корня, полученного с помощью метода половинного деления, с увеличением точности стремиться к нулю немонотонно, пересекает линию заданной точности 2 раза.
		\item Значение корня, полученного с помощью метода простых итераций,с увеличением точности стремиться к нулю монотонно, находится изначально ниже линии заданной точности, то есть метод простых итераций сразу находит корень с довольно большой точностью.
	\end{enumerate}

	\paragraph{}График зависимости модуля значений алгебраической функции от заданной точности показывает, что:
	\begin{enumerate}
		\item Значение корня, полученного с помощью метода fzero, пересекает линию заданной точности 2 раза при $\epsilon=10^{-5}$, но находится в итоге чуть выше неё.
		\item Значения корня, полученного с помощью метода половинного деления, пересекает линию заданной точности 2 раза при $\epsilon=10^{-5}$ , но находится в итоге чуть выше линии заданной точности.
		
	\end{enumerate}
	\paragraph{}График зависимости модуля значений трансцедентной функции от заданной точности показывает, что:
	\begin{enumerate}
		\item Значение корня, полученного с помощью метода fzero, не пересекает линию заданной точности, практически совпадает с ней, но находится в итоге чуть ниже неё.
		\item Значение корня, полученного с помощью метода половинного деления, дважды пересекает заданную линию точности примерно при $\epsilon=10^{-2}$. При точности от $\epsilon=10^{-5}$ до $ 10^{-6}$  не меняется, достигнув порядка $10^{-8}$, то есть
		метод половинного находит корень с довольно большой
		точностью.
		\item Значение корня, полученного с помощью метода простых итераций, после точности $\epsilon=10^{-4}$ не изменяется и один раз пересекает заданную линию точности линию при $\epsilon=10^{-5}$.
	\end{enumerate}
	\section{Выводы}
	\begin{enumerate}
		\item Метод половинного деления более прост и нагляден, при выполнении небольшого количества условий, по сравнению с методом простых итераций, но требует значительно большего количества итераций с увеличением заданной точности.
		\item Метод простых итераций обладает большей скоростью сходиомсти. Данный метод зависит от функции $\varphi$ (т.е. зависит от начальной функции $f$) и параметра q, в свою очередь зависящего от $\varphi$, влияющие на количество итервций. 
	\end{enumerate}
	
\end{document}
